\documentclass[utf8, a4paper, 14pt, russian, oneside]{book}

% Кодировка
\usepackage[T2A]{fontenc}
\usepackage[utf8]{inputenc}
\usepackage[main=russian, english]{babel}

% Пакеты для работы с математикой
\usepackage{amsmath}
\usepackage{amsfonts}
\usepackage{amssymb}

% Вставка изображений
\usepackage{graphicx}

% Пакет для работы с таблицами
\usepackage{tabularx}
\usepackage{booktabs}
\usepackage{longtable}

% Для больших множеств
\usepackage{mathtools}

% Для работы с рисунками
\usepackage{caption}

% Для создания графов в 3 блоке
\usepackage[all]{xy}

% Для специальных символов
\usepackage{textcomp}
\newcommand{\mysec}[1]{
{\center\section*{#1}}
\addcontentsline{toc}{section}{#1}
}

% Команды для настройки содержания
\renewcommand\contentsname{\center{Содержание}} % Вместо оглавления пишется содержание
\addto{\captionsenglish}{\renewcommand{\bibname}{References}}
\begin{document}

\thispagestyle{empty}
~\vspace{-2cm}\setlength{\parindent}{0cm}
\begin{center}
	\includegraphics[scale=1.5]{../include/logo.png}\\[2pt]
	МИНОБРНАУКИ РОССИИ\\
	Федеральное государственное бюджетное образовательное учреждение\\
	высшего профессионального образования\\[5pt]
	\textbf{<<МИРЭА – Российский технологический университет>>}\\[5pt]
	\textbf{\large РТУ МИРЭА}\\[20pt]
	\hrule{}\mbox{}\\[1pt]
	\hrule{}\mbox{}\\[20pt]	
	Институт кибернетики \\ Кафедра <<Инфориационная безопасность>> (БК №252)\\[35pt]
	\textbf{Долгосрочное задание} \\
	по дисциплине: Математическая статистика
\end{center}
	\vspace{4in}
	Студент группы ККСО-01-19:  \qquad \qquad \qquad  \qquad     Колесников А.В.
\vspace{0.6in}
\begin{center}
Москва --- 2021
\end{center}
\newpage

\tableofcontents
\newpage

\mysec{Описание данных}

В качестве данных для анализа выберем оценки критиков, игроков, а такжу общую оценку вместе со всеми отзывами, для игр с сайта metacritic.com. Получим следующую выборку:
\begin{table}[h!]
    \centering
    \begin{tabular}{|c|c|c|c|}
        \hline
        № & $X$ & $Y$ & $Z$ \\ \hline
        1 & 89 & 66 & 89  \\ \hline
        2 & 83 & 80 & 95  \\ \hline
        3 & 82 & 94 & 85  \\ \hline
        4 & 81 & 96 & 93  \\ \hline
        5 & 73 & 77 & 87  \\ \hline
        6 & 95 & 79 & 91  \\ \hline
        7 & 84 & 99 & 87  \\ \hline
        8 & 93 & 80 & 84  \\ \hline
        9 & 84 & 92 & 81  \\ \hline
        10 & 85 & 72 & 93 \\ \hline
        11 & 89 & 76 & 96 \\ \hline
        12 & 76 & 86 & 80 \\ \hline
        13 & 81 & 69 & 83 \\ \hline
        14 & 69 & 68 & 99 \\ \hline
        15 & 91 & 85 & 89 \\ \hline
        16 & 68 & 77 & 87 \\ \hline
        17 & 96 & 77 & 97 \\ \hline
        18 & 76 & 78 & 95 \\ \hline
        19 & 77 & 87 & 90 \\ \hline
        20 & 97 & 93 & 98 \\ \hline
        21 & 79 & 78 & 99 \\ \hline
        22 & 93 & 97 & 98 \\ \hline
        23 & 93 & 96 & 81 \\ \hline
        24 & 91 & 83 & 83 \\ \hline
        25 & 92 & 85 & 92 \\ \hline
        26 & 67 & 78 & 81 \\ \hline
        27 & 89 & 90 & 82 \\ \hline
        28 & 71 & 83 & 86 \\ \hline
        29 & 71 & 79 & 88 \\ \hline
        30 & 98 & 84 & 91 \\ \hline
    \end{tabular} 
\end{table}

\newpage
\mysec{Корреляционный анализ}

\begin{gather*}
    % \xb = 83,77;\  \yb = 82,80;\  \zb = 89,33;
    \xb = \frac{89 + 83 + 82 + \ldots + 71 + 71 + 98}{30} = 83,77; \\
    \yb = \frac{66 + 80 + 94 + \ldots + 83 + 79 + 84}{30} = 82,80; \\
    \zb = \frac{89 + 95 + 85 + \ldots + 86 + 88 + 91}{30} = 89,33; \\
    \Db(x) = S^2(x) = \frac{(89 - 83,77)^2 + \ldots + (98 - 83,77)^2}{30} = 85,45; \\
    \Db(y) = S^2(y) = \frac{(66 - 82,80)^2 + \ldots + (84 - 82,80)^2}{30} = 76,23; \\
    \Db(z) = S^2(z) = \frac{(89 - 89,33)^2 + \ldots + (91 - 89,33)^2}{30} = 35,02; \\
    \sigma(x) = \sqrt{\Db(x)} = 9,24; \sigma(y) = \sqrt{\Db(y)} = 8,73; \sigma(z) = \sqrt{\Db(z)} = 5,92;
\end{gather*}

Найдем выборочные коэффициенты корреляции для каждой пары выборок:
\begin{gather*}
    \r(x,y) = \frac{\cov(x,y)}{\sigma(x)\sigma(y)} = 0,2881; \\
    \r(x, z) = 0,1720; \r(y, z) = -0,1781;
\end{gather*}

Далее найдем условные среднеквадратические отклонения:
\begin{gather*}
    S_{x/y} = \sqrt{S_x \cdot (1 - \r^2(x, y))} = 8,8517; \\
    S_{y/x} = 8,3606; S_{z/x} = 5,8297; S_{z/y} = 5,8234;
\end{gather*}

\newpage
\mysec{Частные коэффициенты корреляции}
Найдем частные коэффициенты корреляции по следующей формуле:
\begin{gather*}
    r_{xy/z} = \frac{
        \r(x,y) - \r(x, z)\r(y,z)
    }{
        \sqrt{1 - \r^2(x,z)}\sqrt{1 - \r^2(y, z)}
    } = 0,3288; \\
    r_{yz/x} = -0,3258; r_{xz/y} = 0,2370;
\end{gather*}

Сопоставим частные коэффициенты корреляции:
\begin{gather*}
    \r(x,y) < r_{xy/z} \Rightarrow \  \text{влияние усиливается}; \\
    \r(x,z) < r_{xz/y} \Rightarrow \  \text{влияние усиливается}; \\
    \r(y,z) > r_{yz/x} \Rightarrow \  \text{влияние ослабевает};
\end{gather*}

\mysec{Условное среднеквадратическое отклонение и множественный коэффициент корреляции}
Вычислим условное среднеквадратическое отклонение по следующей формуле:
\begin{gather*}
    S_{z/xy} = S_{z/x} \sqrt{1 - r^2_{yz/x}} = 5,5116
\end{gather*}
Найдем множественный коэффициент корреляции:
\begin{gather*}
    r_z = \sqrt{1 - \frac{S^2_{z/xy}}{S^2_z}} = 0,3642
\end{gather*}
Проверим полученный множественный коэффициент корреляции через парные коэффициенты корреляции:
\begin{gather*}
    r_z = \sqrt{
        \frac{
            \r^2(x, z) + \r^2(y, z) - 2\r(x,z)\r(y,z)\r(x,y)
            }{
                1 - \r^2(x, y)
        }
    } = 0,3642
\end{gather*}

\mysec{Значимость множественного коэффициента корреляции}
Проверим значимость множественного коэффициента корреляции $H_0: r^2_z = 0$ - множественный коэффициент корреляции незначим.

Вычислим статистику критерия:
\begin{gather*}
    F_{\text{набл}} = \frac{\frac{r_z^2}{2}}{\frac{1-r_z^2}{n-3}} = 2,0643; \\
    F_{\text{табл}}(\alpha=0,05; df_1=2;df_2=n-3) = 3,354; \\
    F_{\text{набл}} < F_{\text{табл}} \Rightarrow H_0\  \text{не отклоняется и}\  r^2_z \  \text{не значим}
\end{gather*}

\mysec{Уравнение регрессии}

Выпишем уравнение регрессии $z$ по $x, y$:
\begin{gather*}
    \overline{z}(x,y) - \overline{\zb} = b_{xz/y}(x - \overline{\xb}) + b_{yz/x}(y - \overline{\yb})
\end{gather*}

Найдем коэффициенты уравнения выше:
\begin{gather*}
    b_{xz/y} = r_{xz/y} \frac{S_{z/y}}{S_{x/y}} = 0,1559; \\ 
    b_{yz/x}=r_{yz/x}\frac{S_{z/x}}{S_{y/x}} = -0,2272; \\
    \overline{z}(x,y) - 89,33 = 0,1559(x - 83,77) - 0,2272(y - 82,80);\\
    \overline{z}(x,y) = 0,1559x - 0,2272y + 95,08
\end{gather*}

\mysec{Доверительные интервалы для коэффициентов множественной регрессии}
\begin{gather*}
    t_1=\frac{
        (b_{xz/y}-\beta_{xz/y})S_{x/y}\sqrt{n-3}
    }{
        S_{z/y}\sqrt{1-r_{zx/y}^2}
    } \\
    t_2=\frac{
        (b_{yz/x}-\beta_{yz/x})S_{y/x}\sqrt{n-3}
    }{
        S_{z/x}\sqrt{1-r_{zy/x}^2}
    }
\end{gather*}
Для построения доверительных интервалов нужно решить следующие
неравенства:
\begin{gather*}
\left|t_1=\frac{(b_{xz/y}-\beta_{xz/y})S_{x/y}\sqrt{n-3}}{S_{z/y}\sqrt{1-r_{zx/y}^2}}\right|<t_{\text{табл}} \\
\left|t_2=\frac{(b_{yz/x}-\beta_{yz/x})S_{y/x}\sqrt{n-3}}{S_{z/x}\sqrt{1-r_{zy/x}^2}}\right|<t_{\text{табл}}
\end{gather*}
, где $t_{\text{табл}}(\alpha = 0,05;df=n-3)=2,052$.

Решив неравенства получим следующие интервалы:
\begin{gather*}
    -0,0965 < \beta_{xz/y} < 0,4083 \\
    -0,4731 < \beta_{yz/x} < 0,0187
\end{gather*}

\end{document}
