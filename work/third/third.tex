\documentclass[utf8, a4paper, 14pt, russian, oneside]{book}

% Кодировка
\usepackage[T2A]{fontenc}
\usepackage[utf8]{inputenc}
\usepackage[main=russian, english]{babel}

% Пакеты для работы с математикой
\usepackage{amsmath}
\usepackage{amsfonts}
\usepackage{amssymb}

% Вставка изображений
\usepackage{graphicx}

% Пакет для работы с таблицами
\usepackage{tabularx}
\usepackage{booktabs}
\usepackage{longtable}

% Для больших множеств
\usepackage{mathtools}

% Для работы с рисунками
\usepackage{caption}

% Для создания графов в 3 блоке
\usepackage[all]{xy}

% Для специальных символов
\usepackage{textcomp}
\newcommand{\mysec}[1]{
{\center\section*{#1}}
\addcontentsline{toc}{section}{#1}
}

% Команды для настройки содержания
\renewcommand\contentsname{\center{Содержание}} % Вместо оглавления пишется содержание
\addto{\captionsenglish}{\renewcommand{\bibname}{References}}
\begin{document}

\thispagestyle{empty}
~\vspace{-2cm}\setlength{\parindent}{0cm}
\begin{center}
	\includegraphics[scale=1.5]{../include/logo.png}\\[2pt]
	МИНОБРНАУКИ РОССИИ\\
	Федеральное государственное бюджетное образовательное учреждение\\
	высшего профессионального образования\\[5pt]
	\textbf{<<МИРЭА – Российский технологический университет>>}\\[5pt]
	\textbf{\large РТУ МИРЭА}\\[20pt]
	\hrule{}\mbox{}\\[1pt]
	\hrule{}\mbox{}\\[20pt]	
	Институт кибернетики \\ Кафедра <<Инфориационная безопасность>> (БК №252)\\[35pt]
	\textbf{Долгосрочное задание} \\
	по дисциплине: Математическая статистика
\end{center}
	\vspace{4in}
	Студент группы ККСО-01-19:  \qquad \qquad \qquad  \qquad     Колесников А.В.
\vspace{0.6in}
\begin{center}
Москва --- 2021
\end{center}
\newpage

\tableofcontents
\newpage
\mysec{Описание данных}

В качестве данных для анализа выберем оценки критиков и игроков на различные игры с сайта metacritic.
Получим следующую выборку:
\begin{table}[h!]
    \centering
    \begin{tabular}{|c|c|c|c|c|c|c|c|c|c|c|c|c|c|c|c|}
        \hline
        № & 1 & 2 & 3 & 4 & 5 & 6 & 7 & 8 & 9 & 10 & 11 & 12 & 13 & 14 & 15 \\ \hline
        $X_i$ & 92 & 88 & 86 & 72 & 78 & 88 & 87 & 86 & 85 & 85 & 85 & 84 & 82 & 77 & 80 \\ \hline
        $Y_i$ & 86 & 91 & 84 & 68 & 67 & 85 & 76 & 79 & 67 & 74 & 62 & 77 & 72 & 63 & 55 \\ \hline
        \multicolumn{16}{|c|}{} \\ \hline
        № & 16 & 17 & 18 & 19 & 20 & 21 & 22 & 23 & 24 & 25 & 26 & 27 & 28 & 29 & 30 \\ \hline
        $X_i$ & 78 & 97 & 97 & 78 & 71 & 96 & 94 & 93 & 95 & 97 & 77 & 84 & 67 & 76 & 78 \\ \hline
        $Y_i$ & 26 & 91 & 88 & 78 & 37 & 88 & 86 & 91 & 83 & 86 & 81 & 82 & 71 & 82 & 85 \\ \hline
    \end{tabular} 
\end{table}

За $X_i$ обозначим оценки критиков, за $Y_i$ - оценки игроков.


\newpage
\mysec{Исследование  корреляционной зависимости}
Для нахождения выборочного коэффицента корреляции необходимо найти следующие характеристики:
\begin{gather*}
    \overline{\xb} = \frac{92 + 88 + 86 + \ldots + 67 + 76 + 78}{30} = 84,4 \\ 
    \Db(x) = \frac{\sum\limits_{i=1}^{n}(x_i - \overline{\xb})^2}{n} \\ 
    \Db(x) = \frac{\left( (92 - 84,4)^2 + \ldots + (78 - 84,4)^2 \right)}{30} = 66,38 \\
    \sigma(x) = \sqrt{\Db(x)} = 8,15 \\
    \overline{\yb} = \frac{86 + 91 + 84 + \ldots + 71 + 82 + 85}{30} = 75,37 \\
    \Db(y) = \frac{\left((86 - 75,37)^2 + \ldots + (85 - 75,37)^2 \right)}{30} = 224,5
    \sigma(y) = \sqrt{\Db(y)} = 14,98 \\
\end{gather*}

Найдем коэффициент ковариации:
\begin{gather*}
    \cov(x,y) = \frac{1}{n} \sum_{i=1}^{n}(x_i - \overline{\xb})(y_i - \overline{\yb}) \\
    \cov(x,y) = 73,1
\end{gather*}

Следовательно, корреляция равна:
\begin{align*}
    \r(x, y) = \frac{\cov(x, y)}{\sigma(x) \cdot \sigma(y)} =  \frac{73,1}{8,15 \cdot 14,98} = 0,599.
\end{align*}

Проверим значимость коэффицента корреляции. $H_0 : \r(x, y) = 0$. Для этого найдем статистику критерия.
\begin{gather*}
    t_{\text{табл}} = \frac{\r(x, y)}{\sqrt{1 - \r^2(x,y)}} \sqrt{n-2} = 3,96 \\
    T_{\text{табл}}(df=n-2)=T_{\text{табл}}(28) = 2,048
\end{gather*}

Получим, что $t_{\text{набл}} > T_{\text{табл}} \Rightarrow H_0$ отклоняется и $\r(x,y)$ значима. 
\newpage

Вычислим коэффициент корреляции Спирмана $r_s$. Проранжируем выборки в порядке неубывания, где $d^2_i$ равен разности между рангами.
\begin{table}[h!]
    \centering
    \begin{tabular}{|c|c|c|c|c|}
        \hline 
        $X$ & $\rang(x)$ & $Y$ & $\rang(Y)$ & $d_i^2$ \\ \hline
        92 &   23 & 86 &   24 &      1 \\ \hline   
        88 & 21,5 & 91 &   29 &  56,25 \\ \hline   
        86 & 18,5 & 84 &   20 &   2,25 \\ \hline   
        72 &    3 & 68 &    8 &     25 \\ \hline   
        78 &  8,5 & 67 &  6,5 &      4 \\ \hline   
        88 & 21,5 & 85 & 21,5 &      0 \\ \hline   
        87 &   20 & 76 &   12 &     64 \\ \hline   
        86 & 18,5 & 79 &   15 &  12,25 \\ \hline   
        85 &   16 & 67 &  6,5 &  90,25 \\ \hline   
        85 &   16 & 74 &   11 &     25 \\ \hline   
        85 &   16 & 62 &    4 &    144 \\ \hline   
        84 & 13,5 & 77 &   13 &   0,25 \\ \hline   
        82 &   12 & 72 &   10 &      4 \\ \hline   
        77 &  5,5 & 63 &    5 &   0,25 \\ \hline   
        80 &   11 & 55 &    3 &     64 \\ \hline   
        78 &  8,5 & 26 &    1 &  56,25 \\ \hline   
        97 &   29 & 91 &   29 &      0 \\ \hline   
        97 &   29 & 88 & 26,5 &   6,25 \\ \hline   
        78 &  8,5 & 78 &   14 &  30,25 \\ \hline   
        71 &    2 & 37 &    2 &      0 \\ \hline   
        96 &   27 & 88 & 26,5 &   0,25 \\ \hline   
        94 &   25 & 86 &   24 &      1 \\ \hline   
        93 &   24 & 91 &   29 &     25 \\ \hline   
        95 &   26 & 83 &   19 &     49 \\ \hline   
        97 &   29 & 86 &   24 &     25 \\ \hline   
        77 &  5,5 & 81 &   16 & 110,25 \\ \hline   
        84 & 13,5 & 82 & 17,5 &     16 \\ \hline   
        67 &    1 & 71 &    9 &     64 \\ \hline   
        76 &    4 & 82 & 17,5 & 182,25 \\ \hline   
        78 &  8,5 & 85 & 21,5 &    169 \\ \hline   
    \end{tabular} 
\end{table}

Найдем коэффициент корреляции Спирмана по следующей формуле:
\begin{align*}
    r_s = 1 - \frac{6\sum d_i^2}{n^3 - n} = 1 - \frac{6 \cdot 1227}{30^3 - 30} = 0,727
\end{align*}

Проверим гипотезу о значимости. $H_0 :r_s = 0$.

Вычислим статистику критерия.
\begin{gather*}
    t_{\text{набл}} = \frac{r_s \sqrt{n-2}}{\sqrt{1 - r_s^2}} = \frac{0,727 \cdot \sqrt{28}}{\sqrt{1 - 0,727^2}} = 5,603 \\ 
    T_{\text{табл}}(df = n - 2)=2,048
\end{gather*}

Получим, что $|t_{\text{набл}}| > T_{\text{табл}} \Rightarrow H_0$ отвергается и коэффициент Спирмана значим.

Вычислим коэффициент корреляции Кэндэлла $r_k$. Проранжируем выборку в порядку неубывания (при равных значениях $X_i$ или $Y_i$ не разделяем ранги между этими значениями).
\begin{table}[h!]
    \small
    \centering
    \begin{tabular}{|c|c|c|c|}
        \hline 
        $X$ & $\rang(X)$ & $Y$ & $\rang(Y)$ \\ \hline
        92 & 23 & 86 & 23 \\ \hline   
        88 & 21 & 91 & 28 \\ \hline   
        86 & 18 & 84 & 20 \\ \hline   
        72 &  3 & 68 &  8 \\ \hline   
        78 &  7 & 67 &  6 \\ \hline   
        88 & 22 & 85 & 21 \\ \hline   
        87 & 20 & 76 & 12 \\ \hline   
        86 & 19 & 79 & 15 \\ \hline   
        85 & 15 & 67 &  7 \\ \hline   
        85 & 16 & 74 & 11 \\ \hline   
        85 & 17 & 62 &  4 \\ \hline   
        84 & 13 & 77 & 13 \\ \hline   
        82 & 12 & 72 & 10 \\ \hline   
        77 &  5 & 63 &  5 \\ \hline   
        80 & 11 & 55 &  3 \\ \hline   
        78 &  8 & 26 &  1 \\ \hline   
        97 & 28 & 91 & 29 \\ \hline   
        97 & 29 & 88 & 26 \\ \hline   
        78 &  9 & 78 & 14 \\ \hline   
        71 &  2 & 37 &  2 \\ \hline   
        96 & 27 & 88 & 27 \\ \hline   
        94 & 25 & 86 & 24 \\ \hline   
        93 & 24 & 91 & 30 \\ \hline   
        95 & 26 & 83 & 19 \\ \hline   
        97 & 30 & 86 & 25 \\ \hline   
        77 &  6 & 81 & 16 \\ \hline   
        84 & 14 & 82 & 17 \\ \hline   
        67 &  1 & 71 &  9 \\ \hline   
        76 &  4 & 82 & 18 \\ \hline  
        78 & 10 & 85 & 22 \\ \hline  
    \end{tabular}
\end{table}
\newpage

Отсортируем полученные ранги относительно $X_i$:
\begin{table}[h!]
    \centering
    \begin{tabular}{|c|c|}
        \hline 
        $\rang(X)$ & $\rang(Y)$ \\ \hline
        1 &  9 \\ \hline 
        2 &  2 \\ \hline 
        3 &  8 \\ \hline 
        4 & 18 \\ \hline 
        5 &  5 \\ \hline 
        6 & 16 \\ \hline 
        7 &  6 \\ \hline 
        8 &  1 \\ \hline 
        9 & 14 \\ \hline 
        10 & 22 \\ \hline 
        11 &  3 \\ \hline 
        12 & 10 \\ \hline 
        13 & 13 \\ \hline 
        14 & 17 \\ \hline 
        15 &  7 \\ \hline 
        16 & 11 \\ \hline 
        17 &  4 \\ \hline 
        18 & 20 \\ \hline 
        19 & 15 \\ \hline 
        20 & 12 \\ \hline 
        21 & 28 \\ \hline 
        22 & 21 \\ \hline 
        23 & 23 \\ \hline 
        24 & 30 \\ \hline 
        25 & 24 \\ \hline 
        26 & 19 \\ \hline 
        27 & 27 \\ \hline 
        28 & 29 \\ \hline 
        29 & 26 \\ \hline 
        30 & 25 \\ \hline 
    \end{tabular}
\end{table}
\newpage

Посчитаем $r_k$:
\begin{gather*}
    r_k = \frac{S}{\tfrac{1}{2}N(N-1)} \\ 
    S = (21 - 8) + (27 -1) + (21 - 6) + \ldots + (0 - 2) + (0 - 1) + (0 - 0) = 229 \\
    r_k = \frac{229}{15 \cdot 29} = 0,5264
\end{gather*}

Проверим значимость коэффицента $H_0: r_k = 0$.

Найдем статистику критерия:
\begin{gather*}
    Z_{\text{набл}} = r_k \sqrt{
        \frac{
            9N(N-1)
        }{
            2(2N+5)
        }
    } = 
    0,5264 \cdot \sqrt{
        \frac{
            9\cdot 30 \cdot 29
        }{
            2 \cdot(2 \cdot 30 + 5)
        }
    } = 4,085\\
    Z_{\text{табл}} = 1,96
\end{gather*}

Поскольку $Z_{text{набл}} > Z_{\text{табл}} \Rightarrow$ коэффициент Кэндэла значим.

\newpage
\mysec{Доверительный интервал для коэффициента корреляции}

Для построения доверительного интервала воспользуемся преобразованием Фишера, которое имеет нормальное распределение с параметрами
$\tfrac{1}{2} \ln\left(\tfrac{\r + 1}{1-\r}\right), \tfrac{1}{\sqrt{n - 3}}$ т.е.:
\begin{gather*}
    z = \frac{1}{2}\ln\left(\frac{\r + 1}{1 - \r}\right) \approx
    N \left(
        \frac{1}{2} \ln
        \left(
            \frac{\rhob + 1}{1 - \rhob}
        \right),
            \frac{1}{\sqrt{n - 3}}
    \right)
\end{gather*}

Тогда при $\alpha = 0,05$:
\begin{gather*}
    P(|z - M(z)| < \epsilon) = 2\Phi(\epsilon\sqrt{(n-3)}) = 0,95 \Rightarrow \epsilon\sqrt{n-3} = 1,96 \Rightarrow \epsilon = \frac{1,96}{\sqrt{n-3}} \\
    |z - M(z)| < \frac{1,96}{\sqrt{n-3}} \Leftrightarrow \\
    \Leftrightarrow
    \frac{1}{2}\ln\left(
        \frac{\r + 1}{1 - \r}
    \right)
    -
    \frac{1,96}{\sqrt{n-3}}
    <
    \frac{1}{2} \ln \left(
        \frac{\rhob + 1}{1 - \rhob}
    \right)
    <
    \frac{1}{2} \ln \left(
        \frac{\r + 1}{1 - \r}
    \right)
    +
    \frac{1,96}{\sqrt{n - 3}}
\end{gather*}

Упрощая полученное выражение получим:
\begin{align*}
    \frac{1 + \r}{1 - \r} \cdot e^{-\tfrac{3,92}{\sqrt{n-3}}}
    <
    \frac{\rhob + 1}{1 - \rhob}
    <
    \frac{1 + \r}{1 - \r} \cdot e^{\tfrac{3,92}{\sqrt{n-3}}}
\end{align*}

Подставим значения $\r$ и $n$ получим неравенство:
\begin{align*}
    1,875 < \frac{\rhob + 1}{1 - \rhob} < 8,479
\end{align*}

После решения неравенства получим, что $0,3 <\rhob < 0,789$ - доверительный интервал для коэффициента корреляции.
\newpage

\mysec{Регрессивный анализ}

\end{document}