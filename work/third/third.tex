\documentclass[utf8, a4paper, 14pt, russian, oneside]{book}

% Кодировка
\usepackage[T2A]{fontenc}
\usepackage[utf8]{inputenc}
\usepackage[main=russian, english]{babel}

% Пакеты для работы с математикой
\usepackage{amsmath}
\usepackage{amsfonts}
\usepackage{amssymb}

% Вставка изображений
\usepackage{graphicx}

% Пакет для работы с таблицами
\usepackage{tabularx}
\usepackage{booktabs}
\usepackage{longtable}

% Для больших множеств
\usepackage{mathtools}

% Для работы с рисунками
\usepackage{caption}

% Для создания графов в 3 блоке
\usepackage[all]{xy}

% Для специальных символов
\usepackage{textcomp}
\newcommand{\mysec}[1]{
{\center\section*{#1}}
\addcontentsline{toc}{section}{#1}
}

% Команды для настройки содержания
\renewcommand\contentsname{\center{Содержание}} % Вместо оглавления пишется содержание
\addto{\captionsenglish}{\renewcommand{\bibname}{References}}
\begin{document}

\thispagestyle{empty}
~\vspace{-2cm}\setlength{\parindent}{0cm}
\begin{center}
	\includegraphics[scale=1.5]{../include/logo.png}\\[2pt]
	МИНОБРНАУКИ РОССИИ\\
	Федеральное государственное бюджетное образовательное учреждение\\
	высшего профессионального образования\\[5pt]
	\textbf{<<МИРЭА – Российский технологический университет>>}\\[5pt]
	\textbf{\large РТУ МИРЭА}\\[20pt]
	\hrule{}\mbox{}\\[1pt]
	\hrule{}\mbox{}\\[20pt]	
	Институт кибернетики \\ Кафедра <<Инфориационная безопасность>> (БК №252)\\[35pt]
	\textbf{Долгосрочное задание} \\
	по дисциплине: Математическая статистика
\end{center}
	\vspace{4in}
	Студент группы ККСО-01-19:  \qquad \qquad \qquad  \qquad     Колесников А.В.
\vspace{0.6in}
\begin{center}
Москва --- 2021
\end{center}
\newpage

\tableofcontents
\newpage
\mysec{Описание данных}

В качестве данных для анализа выберем оценки критиков и игроков на различные игры с сайта metacritic.
Получим следующую выборку:
\begin{table}[h!]
    \centering
    \begin{tabular}{|c|c|c|c|c|c|c|c|c|c|c|c|c|c|c|c|}
        \hline
        № & 1 & 2 & 3 & 4 & 5 & 6 & 7 & 8 & 9 & 10 & 11 & 12 & 13 & 14 & 15 \\ \hline
        $X_i$ & 92 & 88 & 86 & 72 & 78 & 88 & 87 & 86 & 85 & 85 & 85 & 84 & 82 & 77 & 80 \\ \hline
        $Y_i$ & 86 & 91 & 84 & 68 & 67 & 85 & 76 & 79 & 67 & 74 & 62 & 77 & 72 & 63 & 55 \\ \hline
        \multicolumn{16}{|c|}{} \\ \hline
        № & 16 & 17 & 18 & 19 & 20 & 21 & 22 & 23 & 24 & 25 & 26 & 27 & 28 & 29 & 30 \\ \hline
        $X_i$ & 78 & 97 & 97 & 78 & 71 & 96 & 94 & 93 & 95 & 97 & 77 & 84 & 67 & 76 & 78 \\ \hline
        $Y_i$ & 26 & 91 & 88 & 78 & 37 & 88 & 86 & 91 & 83 & 86 & 81 & 82 & 71 & 82 & 85 \\ \hline
    \end{tabular} 
\end{table}

За $X_i$ обозначим оценки критиков, за $Y_i$ - оценки игроков.


\newpage
\mysec{Исследование  корреляционной зависимости}
Для нахождения выборочного коэффицента корреляции необходимо найти следующие характеристики:
\begin{gather*}
    \overline{\xb} = \frac{92 + 88 + 86 + \ldots + 67 + 76 + 78}{30} = 84,4 \\ 
    \Db(x) = \frac{\sum\limits_{i=1}^{n}(x_i - \overline{\xb})^2}{n} \\ 
    \Db(x) = \frac{\left( (92 - 84,4)^2 + \ldots + (78 - 84,4)^2 \right)}{30} = 66,38 \\
    \sigma(x) = \sqrt{\Db(x)} = 8,15 \\
    \overline{\yb} = \frac{86 + 91 + 84 + \ldots + 71 + 82 + 85}{30} = 75,37 \\
    \Db(y) = \frac{\left((86 - 75,37)^2 + \ldots + (85 - 75,37)^2 \right)}{30} = 224,5
    \sigma(y) = \sqrt{\Db(y)} = 14,98 \\
\end{gather*}

Найдем коэффициент ковариации:
\begin{gather*}
    \cov(x,y) = \frac{1}{n} \sum_{i=1}^{n}(x_i - \overline{\xb})(y_i - \overline{\yb}) \\
    \cov(x,y) = 73,1
\end{gather*}

Следовательно, корреляция равна:
\begin{align*}
    \r(x, y) = \frac{\cov(x, y)}{\sigma(x) \cdot \sigma(y)} =  \frac{73,1}{8,15 \cdot 14,98} = 0,599.
\end{align*}

Проверим значимость коэффицента корреляции. $H_0 : \r(x, y) = 0$. Для этого найдем статистику критерия.
\begin{gather*}
    t_{\text{табл}} = \frac{\r(x, y)}{\sqrt{1 - \r^2(x,y)}} \sqrt{n-2} = 3,96 \\
    T_{\text{табл}}(df=n-2)=T_{\text{табл}}(28) = 2,048
\end{gather*}

Получим, что $t_{\text{набл}} > T_{\text{табл}} \Rightarrow H_0$ отклоняется и $\r(x,y)$ значима. 
\newpage

Вычислим коэффициент корреляции Спирмана $r_s$. Проранжируем выборки в порядке неубывания, где $d^2_i$ равен разности между рангами.
\begin{table}[h!]
    \centering
    \begin{tabular}{|c|c|c|c|c|}
        \hline 
        $X$ & $\rang(x)$ & $Y$ & $\rang(Y)$ & $d_i^2$ \\ \hline
        92 &   23 & 86 &   24 &      1 \\ \hline   
        88 & 21,5 & 91 &   29 &  56,25 \\ \hline   
        86 & 18,5 & 84 &   20 &   2,25 \\ \hline   
        72 &    3 & 68 &    8 &     25 \\ \hline   
        78 &  8,5 & 67 &  6,5 &      4 \\ \hline   
        88 & 21,5 & 85 & 21,5 &      0 \\ \hline   
        87 &   20 & 76 &   12 &     64 \\ \hline   
        86 & 18,5 & 79 &   15 &  12,25 \\ \hline   
        85 &   16 & 67 &  6,5 &  90,25 \\ \hline   
        85 &   16 & 74 &   11 &     25 \\ \hline   
        85 &   16 & 62 &    4 &    144 \\ \hline   
        84 & 13,5 & 77 &   13 &   0,25 \\ \hline   
        82 &   12 & 72 &   10 &      4 \\ \hline   
        77 &  5,5 & 63 &    5 &   0,25 \\ \hline   
        80 &   11 & 55 &    3 &     64 \\ \hline   
        78 &  8,5 & 26 &    1 &  56,25 \\ \hline   
        97 &   29 & 91 &   29 &      0 \\ \hline   
        97 &   29 & 88 & 26,5 &   6,25 \\ \hline   
        78 &  8,5 & 78 &   14 &  30,25 \\ \hline   
        71 &    2 & 37 &    2 &      0 \\ \hline   
        96 &   27 & 88 & 26,5 &   0,25 \\ \hline   
        94 &   25 & 86 &   24 &      1 \\ \hline   
        93 &   24 & 91 &   29 &     25 \\ \hline   
        95 &   26 & 83 &   19 &     49 \\ \hline   
        97 &   29 & 86 &   24 &     25 \\ \hline   
        77 &  5,5 & 81 &   16 & 110,25 \\ \hline   
        84 & 13,5 & 82 & 17,5 &     16 \\ \hline   
        67 &    1 & 71 &    9 &     64 \\ \hline   
        76 &    4 & 82 & 17,5 & 182,25 \\ \hline   
        78 &  8,5 & 85 & 21,5 &    169 \\ \hline   
    \end{tabular} 
\end{table}

Найдем коэффициент корреляции Спирмана по следующей формуле:
\begin{align*}
    r_s = 1 - \frac{6\sum d_i^2}{n^3 - n} = 1 - \frac{6 \cdot 1227}{30^3 - 30} = 0,727
\end{align*}

Проверим гипотезу о значимости. $H_0 :r_s = 0$.

Вычислим статистику критерия.
\begin{gather*}
    t_{\text{набл}} = \frac{r_s \sqrt{n-2}}{\sqrt{1 - r_s^2}} = \frac{0,727 \cdot \sqrt{28}}{\sqrt{1 - 0,727^2}} = 5,603 \\ 
    T_{\text{табл}}(df = n - 2)=2,048
\end{gather*}

Получим, что $|t_{\text{набл}}| > T_{\text{табл}} \Rightarrow H_0$ отвергается и коэффициент Спирмана значим.

Вычислим коэффициент корреляции Кэндэлла $r_k$. Проранжируем выборку в порядку неубывания (при равных значениях $X_i$ или $Y_i$ не разделяем ранги между этими значениями).
\begin{table}[h!]
    \small
    \centering
    \begin{tabular}{|c|c|c|c|}
        \hline 
        $X$ & $\rang(X)$ & $Y$ & $\rang(Y)$ \\ \hline
        92 & 23 & 86 & 23 \\ \hline   
        88 & 21 & 91 & 28 \\ \hline   
        86 & 18 & 84 & 20 \\ \hline   
        72 &  3 & 68 &  8 \\ \hline   
        78 &  7 & 67 &  6 \\ \hline   
        88 & 22 & 85 & 21 \\ \hline   
        87 & 20 & 76 & 12 \\ \hline   
        86 & 19 & 79 & 15 \\ \hline   
        85 & 15 & 67 &  7 \\ \hline   
        85 & 16 & 74 & 11 \\ \hline   
        85 & 17 & 62 &  4 \\ \hline   
        84 & 13 & 77 & 13 \\ \hline   
        82 & 12 & 72 & 10 \\ \hline   
        77 &  5 & 63 &  5 \\ \hline   
        80 & 11 & 55 &  3 \\ \hline   
        78 &  8 & 26 &  1 \\ \hline   
        97 & 28 & 91 & 29 \\ \hline   
        97 & 29 & 88 & 26 \\ \hline   
        78 &  9 & 78 & 14 \\ \hline   
        71 &  2 & 37 &  2 \\ \hline   
        96 & 27 & 88 & 27 \\ \hline   
        94 & 25 & 86 & 24 \\ \hline   
        93 & 24 & 91 & 30 \\ \hline   
        95 & 26 & 83 & 19 \\ \hline   
        97 & 30 & 86 & 25 \\ \hline   
        77 &  6 & 81 & 16 \\ \hline   
        84 & 14 & 82 & 17 \\ \hline   
        67 &  1 & 71 &  9 \\ \hline   
        76 &  4 & 82 & 18 \\ \hline  
        78 & 10 & 85 & 22 \\ \hline  
    \end{tabular}
\end{table}
\newpage

Отсортируем полученные ранги относительно $X_i$:
\begin{table}[h!]
    \centering
    \begin{tabular}{|c|c|}
        \hline 
        $\rang(X)$ & $\rang(Y)$ \\ \hline
        1 &  9 \\ \hline 
        2 &  2 \\ \hline 
        3 &  8 \\ \hline 
        4 & 18 \\ \hline 
        5 &  5 \\ \hline 
        6 & 16 \\ \hline 
        7 &  6 \\ \hline 
        8 &  1 \\ \hline 
        9 & 14 \\ \hline 
        10 & 22 \\ \hline 
        11 &  3 \\ \hline 
        12 & 10 \\ \hline 
        13 & 13 \\ \hline 
        14 & 17 \\ \hline 
        15 &  7 \\ \hline 
        16 & 11 \\ \hline 
        17 &  4 \\ \hline 
        18 & 20 \\ \hline 
        19 & 15 \\ \hline 
        20 & 12 \\ \hline 
        21 & 28 \\ \hline 
        22 & 21 \\ \hline 
        23 & 23 \\ \hline 
        24 & 30 \\ \hline 
        25 & 24 \\ \hline 
        26 & 19 \\ \hline 
        27 & 27 \\ \hline 
        28 & 29 \\ \hline 
        29 & 26 \\ \hline 
        30 & 25 \\ \hline 
    \end{tabular}
\end{table}
\newpage

Посчитаем $r_k$:
\begin{gather*}
    r_k = \frac{S}{\tfrac{1}{2}N(N-1)} \\ 
    S = (21 - 8) + (27 -1) + (21 - 6) + \ldots + (0 - 2) + (0 - 1) + (0 - 0) = 229 \\
    r_k = \frac{229}{15 \cdot 29} = 0,5264
\end{gather*}

Проверим значимость коэффицента $H_0: r_k = 0$.

Найдем статистику критерия:
\begin{gather*}
    Z_{\text{набл}} = r_k \sqrt{
        \frac{
            9N(N-1)
        }{
            2(2N+5)
        }
    } = 
    0,5264 \cdot \sqrt{
        \frac{
            9\cdot 30 \cdot 29
        }{
            2 \cdot(2 \cdot 30 + 5)
        }
    } = 4,085\\
    Z_{\text{табл}} = 1,96
\end{gather*}

Поскольку $Z_{text{набл}} > Z_{\text{табл}} \Rightarrow$ коэффициент Кэндэла значим.

\newpage
\mysec{Доверительный интервал для коэффициента корреляции}

Для построения доверительного интервала воспользуемся преобразованием Фишера, которое имеет нормальное распределение с параметрами
$\tfrac{1}{2} \ln\left(\tfrac{\r + 1}{1-\r}\right), \tfrac{1}{\sqrt{n - 3}}$ т.е.:
\begin{gather*}
    z = \frac{1}{2}\ln\left(\frac{\r + 1}{1 - \r}\right) \approx
    N \left(
        \frac{1}{2} \ln
        \left(
            \frac{\rhob + 1}{1 - \rhob}
        \right),
            \frac{1}{\sqrt{n - 3}}
    \right)
\end{gather*}

Тогда при $\alpha = 0,05$:
\begin{gather*}
    P(|z - M(z)| < \varepsilon) = 2\Phi(\varepsilon\sqrt{(n-3)}) = 0,95 \Rightarrow \varepsilon\sqrt{n-3} = 1,96 \Rightarrow \varepsilon = \frac{1,96}{\sqrt{n-3}} \\
    |z - M(z)| < \frac{1,96}{\sqrt{n-3}} \Leftrightarrow \\
    \Leftrightarrow
    \frac{1}{2}\ln\left(
        \frac{\r + 1}{1 - \r}
    \right)
    -
    \frac{1,96}{\sqrt{n-3}}
    <
    \frac{1}{2} \ln \left(
        \frac{\rhob + 1}{1 - \rhob}
    \right)
    <
    \frac{1}{2} \ln \left(
        \frac{\r + 1}{1 - \r}
    \right)
    +
    \frac{1,96}{\sqrt{n - 3}}
\end{gather*}

Упрощая полученное выражение получим:
\begin{align*}
    \frac{1 + \r}{1 - \r} \cdot e^{-\tfrac{3,92}{\sqrt{n-3}}}
    <
    \frac{\rhob + 1}{1 - \rhob}
    <
    \frac{1 + \r}{1 - \r} \cdot e^{\tfrac{3,92}{\sqrt{n-3}}}
\end{align*}

Подставим значения $\r$ и $n$ получим неравенство:
\begin{align*}
    1,875 < \frac{\rhob + 1}{1 - \rhob} < 8,479
\end{align*}

После решения неравенства получим, что $0,3 <\rhob < 0,789$ - доверительный интервал для коэффициента корреляции.
\newpage

\mysec{Регрессивный анализ}
Регрессией называется функция проивзольного вида, которая определена как условное математическое ожидание $Y$ при условии $X=x$, т.е.
\begin{align*}
    E(Y|X) = u(x)
\end{align*}

\mysubsec{Линейная регрессия}
Линейное уравнение регрессии в общем вдие записывается следующим образом:
\begin{align*}
    \overline{\yx} = kx + b
\end{align*}

Составим функцию $\Upphi(k, b) = \sum_{i=1}^{n} (y_i - kx_i - b)^2$ и найдем частные производные этой функции:
\begin{align*}
    \begin{cases}
        \der{\Upphi}{k} = 0 \\
        \der{\Upphi}{b} = 0 \\
    \end{cases}
    \Leftrightarrow
    \begin{cases}
        2 \sum (y_i - kx_xi - b)(-x_i) = 0 \\
        2 \sum (y_i - kx_xi - b)(-1) = 0
    \end{cases}
    \Leftrightarrow
    \begin{cases}
        k\sum x_i^2 +b \sum x_i = \sum y_i x_i \\
        k\sum x_i +b n = \sum y_i
    \end{cases}
\end{align*}

Найдем оценки параметров:
\begin{gather*}
    b = \frac{\sum y_i}{n} - k \sum \frac{x_i}{n} = \overline{y} - k\overline{x} \\
    k\sum x_i^2 + \overline{y} - k \overline{x} \sum x_i = \sum y_i x_i \\
    k \left( \sum x_i^2 - \overline{x} \sum x_i \right) = \sum y_ix_i - \overline{y}\sum x_i \\
    k = \frac{
        \sum y_ix_i - \overline{y}\sum x_i
    }{
        \sum x_i^2 - \overline{x} \sum x_i
    } = 
    \frac{\cov{x, y}}{\Db(x)} \\
    b = \overline{y} - \frac{\cov(x,y)}{\Db(x)}\overline{x}
\end{gather*}

Таким образом коэффициенты равны:
\begin{align*}
    b = -17,668; k =  1,102   
\end{align*}

И общее уравнение линейной регрессии будет выглядеть:
\begin{align*}
    \overline{\yx} = 1,102x - 17,668
\end{align*}

Теперь Вычислим ошибки модели по формуле $\varepsilon_i = y_i - kx_i - b$. В качестве примера посчитаем первые два значения:
\begin{gather*}
    \varepsilon_1 = 86 - 1,102 \cdot 92 + 17,668 = 2,30 \\
    \varepsilon_2 = 91 - 1,102 \cdot 88 + 17,668 = 11,70
\end{gather*}

Построим таблицу значений:
\begin{table}[h!]
    \centering
    \begin{tabular}{|c|c|c|c|c|c|c|}
        \hline
        $\varepsilon_i$ & Значение & $\varepsilon_i$ & Значение & $\varepsilon_i$ & Значение  \\ \hline
        $\varepsilon_1$ & 2,30  & $\varepsilon_{11}$ & -13,99    &$\varepsilon_{21}$ & -0,11  \\ \hline
        $\varepsilon_2$ & 11,70  & $\varepsilon_{12}$ & 2,11     & $\varepsilon_{22}$ & 0,09  \\ \hline
        $\varepsilon_3$ & 6,91  & $\varepsilon_{13}$ & -0,69     & $\varepsilon_{23}$ & 6,19  \\ \hline
        $\varepsilon_4$ & 6,33  & $\varepsilon_{14}$ & -4,18     &$\varepsilon_{24}$ & -4,01  \\ \hline
        $\varepsilon_5$ & -1,28  & $\varepsilon_{15}$ & -15,48   &$\varepsilon_{25}$ & -3,21  \\ \hline
        $\varepsilon_6$ & 5,70  & $\varepsilon_{16}$ & -42,28    &$\varepsilon_{26}$ & 13,82  \\ \hline
        $\varepsilon_7$ & -2,19  & $\varepsilon_{17}$ & 1,79     & $\varepsilon_{27}$ & 7,11  \\ \hline
        $\varepsilon_8$ & 1,91  & $\varepsilon_{18}$ & -1,21     &$\varepsilon_{28}$ & 14,84  \\ \hline
        $\varepsilon_9$ & -8,99 &  $\varepsilon_{19}$ & 9,72     &$\varepsilon_{29}$ & 15,93  \\ \hline
        $\varepsilon_{10}$ & -1,99 & $\varepsilon_{20}$ & -23,56 &$\varepsilon_{30}$ & 16,72  \\ \hline
    \end{tabular} 
\end{table}

Далее вычислим коэффициент детерминации по формуле:
\begin{align*}
    R^2 = 1 - \frac{\sum(y_i - kx_i - b)^2}{\sum(y_i - \overline{y})^2} = 1 - \frac{\tfrac{1}{2}\sum \varepsilon_i^2}{\Db(y)} = 0,359
\end{align*}

Проверим значимость. $H_0 : R^2 = 0$ (уравнение регрессии незначимо). Для этого вычислим статистику критерия:
\begin{gather*}
    F_{\text{набл}} = \frac{R^2}{1-R^2}(n-2) = 15,68 \\
    F_{\text{табл}}  =(df_1 = 1; df_2 = n-2) = 4,196 \\
    F_{\text{набл}} > F_{\text{табл}} \Rightarrow H_0\  \text{отклоняется}.
\end{gather*}
\newpage

\mysubsec{Квадратичная регрессия}

Построим уравнение квадратичной регрессии $\overline{\yx} = a_2x^2 + a_1x+a_0$.
\begin{gather*}
    \Upphi(a_0, a_1, a_2) = \sum_{i=1}^2 (y_i - a_0 - a_1x_i - a_2x_i^2)^2 \rightarrow min \\
    \begin{cases}
        \der{\Upphi}{a_2} = 0, \\
        \der{\Upphi}{a_1} = 0, \\
        \der{\Upphi}{a_0} = 0
    \end{cases}
    \Leftrightarrow
    \begin{cases}
        a_2 \sum\limits_{i=1}^{n} x_i^4 + a_1 \sum\limits_{i=1}^{n} x_i^3 + a_0 \sum\limits_{i=1}^{n} x_i^2 = \sum\limits_{i=1}^n y_i x_i^2, \\
        a_2 \sum\limits_{i=1}^{n} x_i^3 + a_1 \sum\limits_{i=1}^{n} x_i^2 + a_0 \sum\limits_{i=1}^{n} x_i = \sum\limits_{i=1}^n y_i x_i, \\
        a_2 \sum\limits_{i=1}^{n} x_i^2 + a_1 \sum\limits_{i=1}^{n} x_i + a_0 n = \sum\limits_{i=1}^n y_i
    \end{cases}
    \Leftrightarrow \\
    \Leftrightarrow
    \begin{cases}
        a_2 \cdot 1609636445 + a_1 \cdot 18560641 + a_0 \cdot 215861 = 16639430, \\
        a_2 \cdot 18560641 + a_1 \cdot 215861 + a_0 \cdot 2533 = 193098, \\
        a_2 \cdot 215861 + a_1 \cdot 2533 + a_0 \cdot 30 = 2261
    \end{cases}
\end{gather*}

Решив систему, получим следующий результат:
\begin{align*}
    \begin{cases}
        a_2 = 0,01165032285, \\
        a_1 = -0,8567129022, \\
        a_0 = 63,87344805
    \end{cases}
\end{align*}

Тогда уравнение квадратичной регрессии имеет вид:
\begin{align*}
    \overline{\yx} = 0,0116 x^2 - 0,8567 x + 63,8734
\end{align*}

Теперь подсчитаем коэффициент детерминации, перед этим подсчитав ошибки модели. Посчитаем несколько значений в качестве примера:
\begin{gather*}
    \varepsilon_i = y_i - 0,0116x_i^2 + 0,8567x_i - 63,8734 \\
    \varepsilon_1 = 86 - 0,0116 \cdot 92^2  + 0,8567 \cdot 92 - 63,8734 = 2,33\\
    \varepsilon_2 = 91 - 0,0116 \cdot 88 + 0,8567 \cdot 88 - 63,8734 = 12,30
\end{gather*}

\newpage

Построим таблицу значений:
\begin{table}[h!]
    \centering
    \begin{tabular}{|c|c|c|c|c|c|}
        \hline
        $\varepsilon_i$ & Значение & $\varepsilon_i$ & Значение & $\varepsilon_i$ & Значение \\ \hline
        $\varepsilon_{1}$  & 2,33  & $\varepsilon_{11}$ & -13,23 & $\varepsilon_{21}$ & -1,00 \\ \hline
        $\varepsilon_{2}$  & 12,30 & $\varepsilon_{12}$ & 2,88   & $\varepsilon_{22}$ & -0,28 \\ \hline
        $\varepsilon_{3}$  & 7,64  & $\varepsilon_{13}$ & 0,04   & $\varepsilon_{23}$ & 6,04 \\ \hline
        $\varepsilon_{4}$  & 5,41  & $\varepsilon_{14}$ & -3,98  & $\varepsilon_{24}$ & -4,63 \\ \hline
        $\varepsilon_{5}$  & -0,93 & $\varepsilon_{15}$ & -14,89 & $\varepsilon_{25}$ & -4,39 \\ \hline
        $\varepsilon_{6}$  & 6,30  & $\varepsilon_{16}$ & -41,93 & $\varepsilon_{26}$ & 14,02 \\ \hline
        $\varepsilon_{7}$  & -1,52 & $\varepsilon_{17}$ & 0,61   & $\varepsilon_{27}$ & 7,88 \\ \hline
        $\varepsilon_{8}$  & 2,64  & $\varepsilon_{18}$ & -2,39  & $\varepsilon_{28}$ & 12,23 \\ \hline
        $\varepsilon_{9}$  & -8,22 & $\varepsilon_{19}$ & 10,07  & $\varepsilon_{29}$ & 15,94 \\ \hline
        $\varepsilon_{10}$ & -1,23 & $\varepsilon_{20}$ & -24,78 & $\varepsilon_{30}$ & 17,07 \\ \hline
    \end{tabular}
\end{table}
\begin{align*}
    R^2 = 1 - \frac{
        \tfrac{1}{n} \sum\limits_{i=1}^n \varepsilon_i^2
    }{\Db(y)}
    =
    1 - \frac{4295,9505}{30 \cdot 224,5} = 0,3621
\end{align*}

Проверим гипотезу $H_0$ о значимости $R^2$:
\begin{gather*}
    F_{\text{набл}} = \frac{R^2}{1-R^2} \left( \frac{n - m - 1}{m} \right) \\
    F_{\text{набл}} = \frac{0,3621}{1 - 0,3621} \cdot \frac{30 - 2 -1}{2} = 7,665 \\
    F_{\text{табл}} = 3,354
\end{gather*}

Получим, что $F_{\text{набл}} > F_{\text{табл}} \Rightarrow H_0$ отклоняется. 

\newpage

\mysubsec{Кубическая регрессия}
Формула кубической регрессии имеет следующий вид: $\overline{\yx} = a_3x^3 + a_2x^2 + a_1x + a_0$.

\begin{gather*}
    \begin{cases}
        a_3\sum\limits_{i=1}^n x_i^6 + a_2 \sum\limits_{i=1}^n x_i^5 + a_1 \sum\limits_{i=1}^n x_i^4 + a_0 \sum\limits_{i=1}^n x_i^3 = \sum\limits_{i=1}^n y_ix_i^3, \\
        a_3\sum\limits_{i=1}^n x_i^5 + a_2 \sum\limits_{i=1}^n x_i^4 + a_1 \sum\limits_{i=1}^n x_i^3 + a_0 \sum\limits_{i=1}^n x_i^2 = \sum\limits_{i=1}^n y_ix_i^2, \\
        a_3\sum\limits_{i=1}^n x_i^4 + a_2 \sum\limits_{i=1}^n x_i^3 + a_1 \sum\limits_{i=1}^n x_i^2 + a_0 \sum\limits_{i=1}^n x_i   = \sum\limits_{i=1}^n y_ix_i, \\
        a_3\sum\limits_{i=1}^n x_i^3 + a_2 \sum\limits_{i=1}^n x_i^2 + a_1 \sum\limits_{i=1}^n x_i   + a_0 n     = \sum\limits_{i=1}^n y_i
    \end{cases}
    \Leftrightarrow \\ 
    \Leftrightarrow
    \small
    \begin{cases}
        a_3 \cdot 12399960614501 + a_2 \cdot 140734344073 + a_1 \cdot 1609636445 + a_0 \cdot 18560641 = 1446099192, \\
        a_3 \cdot 140734344073   + a_2 \cdot 1609636445   + a_1 \cdot 18560641   + a_0 \cdot 215861   = 16639430, \\
        a_3 \cdot 1609636445     + a_2 \cdot 18560641     + a_1 \cdot 215861     + a_0 \cdot 2533     = 193098, \\
        a_3 \cdot 18560641       + a_2 \cdot 215861       + a_1 \cdot 2533       + a_0 \cdot 30       = 2261
    \end{cases}
\end{gather*}

Решив систему уравнений получаем следующие коэффициенты:
\begin{align*}
    \begin{cases}
        a_3 = -0,003749754152, \\
        a_2 = 0,9439807581, \\
        a_1 = -77,56771874, \\
        a_0 = 2152,30806
    \end{cases}
\end{align*}

Тогда кубическое уравнение регрессии примет вид:
\begin{align*}
    \overline{\yx} = -0,0037 x^3 + 0,9440 x^2 - 77,5677 x + 2142,3080
\end{align*}

\newpage

Найдем ошибки Можеле по формуле $\varepsilon_i = y_i - a_3 x^3 - a_2 x^2 - a_1 x - a_0$.
Составим таблицу значений.
\begin{table}[h!]
    \centering
    \begin{tabular}{|c|c|c|c|c|c|}
        \hline
        $\varepsilon_i$ & Значение & $\varepsilon_i$ & Значение &$\varepsilon_i$ & Значение \\ \hline
        $\varepsilon_{1}$  & -0,04 & $\varepsilon_{11}$ & -14,49 & $\varepsilon_{21}$ & 0 \\ \hline
        $\varepsilon_{2}$  & 9,82  & $\varepsilon_{12}$ & 2,14   & $\varepsilon_{22}$ & -1,47 \\ \hline
        $\varepsilon_{3}$  & 5,88  & $\varepsilon_{13}$ & 0,41   & $\varepsilon_{23}$ & 4,14 \\ \hline
        $\varepsilon_{4}$  & 6,56  & $\varepsilon_{14}$ & -1,57  & $\varepsilon_{24}$ & -4,85 \\ \hline
        $\varepsilon_{5}$  & 1,25  & $\varepsilon_{15}$ & -13,49 & $\varepsilon_{25}$ & -1,85 \\ \hline
        $\varepsilon_{6}$  & 3,82  & $\varepsilon_{16}$ & -39,75 & $\varepsilon_{26}$ & 16,43 \\ \hline
        $\varepsilon_{7}$  & -3,68 & $\varepsilon_{17}$ & 3,14   & $\varepsilon_{27}$ & 7,15 \\ \hline
        $\varepsilon_{8}$  & 0,88  & $\varepsilon_{18}$ & 0,14   & $\varepsilon_{28}$ & 5,99 \\ \hline
        $\varepsilon_{9}$  & -9,49 & $\varepsilon_{19}$ & 12,25  & $\varepsilon_{29}$ & 18,46 \\ \hline
        $\varepsilon_{10}$ & -2,49 & $\varepsilon_{20}$ & -24,53 & $\varepsilon_{30}$ & 19,25 \\ \hline
    \end{tabular}
\end{table}

Детерминация равна:
\begin{gather*}
    R^2 = 1 - \frac{
        \tfrac{1}{n} \sum\limits_{i=1}^{n} \varepsilon_i^2
    }{\Db(y)} = 0,3828
\end{gather*}

Проверим гипотезу о значимости уравнения регрессии $H_0: R^2 = 0$.
Вычислим статистику критерия:
\begin{gather*}
    F_{\text{набл}} = \frac{R^2}{1-R^2} \cdot \frac{n - m - 1}{m} = \frac{0,3828}{1 - 0,3828} \cdot \frac{30 - 3 - 1}{3} = 5,38\\
    F_{\text{табл}}(df_1 = m; df_2 =n-m-1)=2,98
\end{gather*}

Поскольку $F_{\text{набл}} > F_{\text{табл}} \Rightarrow H_0$ отвергается и $R^2$ значим.

\newpage
\mysubsec{Показательная регрессия}

Уравнение показательной регрессии в общем виде выглядит как $\overline{\yx} = ba^x$. Применим к этому уравнению метод линеризации:
\begin{align*}
    \ln(\overline{\yx}) = \ln(b) + x\ln(a)
\end{align*}

Сделаем замену $z = \ln(y); \hat{b} = \exp{(\hat{B})}; \hat{a} = \exp{(\hat{k})}$ и получим:
\begin{align*}
    \overline{\zx} = \hat{k}x + \hat{B}
\end{align*}

Рассмотрим наблюдения вида $(x_1, \ln(y_1)); \ldots; (x_n; \ln(y_n))$ и используем формулу:
\begin{align*}
    \frac{
        \overline{\yx} - \overline{\yb}
    }{
        \sb(y)
    } = \r(x,y) \frac{x - \overline{\xb}}{\sb(x)}
\end{align*}

Сгруппировав возле $\overline{\yx}$ и $x$ коэффициенты получим соответствующие $\hat{k}$ и $\hat{B}$.
\begin{align*}
    \overline{\zx} = \frac{
        \r(x, \ln(y)) \sb(\ln(y))
    }{
        \sb(x)
    }x
    -
    \frac{
        \sb(\ln(y))
    }{
        \sb(x)
    }\overline{\xb}
    +
    \overline{\ln{(\yb)}}
\end{align*}

Пересчитаем коэффициенты корреляции, среднее квадратичное отклонение и среднее выборочное значение $y_i$.
\begin{gather*}
    \overline{\ln(\yb)} = \frac{
        \sum\limits_{i=1}^n n_i \ln(y_i)
    }{30} = 4,29 \\
    \sb(\ln(y)) = 0,265 \\
    \r(x, \ln(y)) = \frac{\cov{(x, \ln(y))}}{\sigma(x)\sigma(\ln(y))} = 
    \frac{
        \sum\limits_{i=1}^n (x_i - \overline{\xb})(\ln(y_i) - \overline{\ln(\yb)})
    }{
        n\sigma(x)\sigma(\ln(y))
    } = 0,523
\end{gather*}

Получим следующее уравнение:
\begin{align*}
    \overline{\zx} = \frac{0,523 \cdot 0,265}{8,15}x - \frac{0,265 \cdot 0,523 \cdot 84,43}{8,15} + 4,29 = 0,017x + 2,8575
\end{align*}

Произведем обратную замену и получим следующие коэффициенты показательной регрессии:
\begin{align*}
    \hat{a} = \exp(0,017) = 1,017; \hat{b} = \exp(2,8575) = 17,418
\end{align*}

Значит уравнение показательной регрессии имеет вид:
\begin{align*}
    \overline{\yx} = 17,418 \cdot 1,017^x
\end{align*}

\newpage
Найдем ошибки уравнения регрессии по формуле $\varepsilon_i = y_i - ba^{x_i}$.
\begin{table}[h!]
    \centering
    \begin{tabular}{|c|c|c|c|c|c|}
        \hline
        $\varepsilon_i$ & Значение & $\varepsilon_i$ & Значение &$\varepsilon_i$ & Значение \\ \hline
        $\varepsilon_{1}$  & 2,70  & $\varepsilon_{11}$ & -11,95 & $\varepsilon_{21}$ & -1,17 \\ \hline
        $\varepsilon_{2}$  & 13,18 & $\varepsilon_{12}$ & 4,30   & $\varepsilon_{22}$ & -0,18 \\ \hline
        $\varepsilon_{3}$  & 8,78  & $\varepsilon_{13}$ & 1,73   & $\varepsilon_{23}$ & 6,27 \\ \hline
        $\varepsilon_{4}$  & 8,72  & $\varepsilon_{14}$ & -1,54  & $\varepsilon_{24}$ & -4,66 \\ \hline
        $\varepsilon_{5}$  & 1,35  & $\varepsilon_{15}$ & -12,92 & $\varepsilon_{25}$ & -4,70 \\ \hline
        $\varepsilon_{6}$  & 7,18  & $\varepsilon_{16}$ & -39,70 & $\varepsilon_{26}$ & 16,46 \\ \hline
        $\varepsilon_{7}$  & -0,51 & $\varepsilon_{17}$ & 0,30   & $\varepsilon_{27}$ & 9,30 \\ \hline
        $\varepsilon_{8}$  & 3,78  & $\varepsilon_{18}$ & -2,70  & $\varepsilon_{28}$ & 16,55 \\ \hline
        $\varepsilon_{9}$  & -6,95 & $\varepsilon_{19}$ & 12,35  & $\varepsilon_{29}$ & 18,55 \\ \hline
        $\varepsilon_{10}$ & 0,05  & $\varepsilon_{20}$ & -21,28 & $\varepsilon_{30}$ & 19,35 \\ \hline
    \end{tabular}
\end{table}
\begin{align*}
    R^2 = 1 - \frac{
        \tfrac{1}{n}\sum\limits_{i=1}^n \varepsilon_i^2
    }{\Db(y)} = 146,76
\end{align*}

\newpage
\mysubsec{Степенная регрессия}

Уравнение степенной регрессии в общем виде: $\overline{\yx} = bx^a$.
Применим к этому уравнению метод линеризации:
\begin{align*}
    \ln(\overline{\yx}) = \ln b + a \ln x
\end{align*}

Сделаем замену $z = \ln y; \hat{b} = \exp{(\hat{B})}; \ln x = u$:
\begin{align*}
    \overline{\zx} = \hat{a}u + \hat{B}
\end{align*}

Рассмотрим наблюдения вида $\ln(x_1, \ln y_1); \ldots; (\ln x_n; \ln y_n)$ и используем следующую формулу:
\begin{align*}
    \frac{
        \overline{\yx} - \overline{\yb}
    }{\sb(y)}
    =
    \r(x,y) \frac{x - \overline{\xb}}{\sb(x)}
\end{align*}

Пересчитаем коэффициенты корреляции, среднее квадратичное отклонение и среднее выборочное значение $y_i$ и $x_i$:
\begin{gather*}
    \overline{\ln \yb} = \frac{
        \sum\limits_{i=1}^n n_i \ln y_i
    }{30} = 4,29 \\
    \overline{\ln \xb} = \frac{
        \sum\limits_{i=1}^n n_i \ln x_i
    }{30} = 4,43 \\
    \sb(\ln y) = 0,265 \\
    \sb(\ln x) = 0,098 \\
    \r(\ln x, \ln y) = \frac{\cov(\ln x, \ln y)}{\sigma(\ln x)\sigma(\ln y)}
    =
    \frac{
        \sum\limits_{i=1}^n (\ln x_i - \overline{\ln \xb})(\ln y_i - \overline{\ln \yb})
    }{
        n \sigma(\ln x)\sigma(\ln y)
    }=
    0,520
\end{gather*}

Получим следующее уравнение:
\begin{align*}
    \zx = \frac{0,520 \cdot 0,265}{0,098}x - \frac{0,265 \cdot 0,520 \cdot 4,43}{0,098} + 4,29 = 1,4099x - 1,9537
\end{align*}

Произведем обратную замену и получим следующие коэффициенты степенной регрессии:
\begin{align*}
    \hat{a} = 1,4099; \hat{b} = \exp(-1,9537) = 0,1418
\end{align*}

Значит, уравнение степенной регрессии примет вид:
\begin{align*}
    \overline{\yx} = 0,1418 \cdot x^{4,0954}
\end{align*}
\newpage

Найдем ошибки уравнения регрессии по формуле $\varepsilon_i = y_i - bx_i^a$ и составим таблицу значений:
\begin{table}[h!]
    \centering
    \begin{tabular}{|c|c|c|c|c|c|}
        \hline
        $\varepsilon_i$ & Значение & $\varepsilon_i$ & Значение &$\varepsilon_i$ & Значение \\ \hline
        $\varepsilon_{1}$  & 2,78  & $\varepsilon_{11}$ & -12,43 & $\varepsilon_{21}$ & -0,36 \\ \hline
        $\varepsilon_{2}$  & 12,84 & $\varepsilon_{12}$ & 3,80   & $\varepsilon_{22}$ & 0,22  \\ \hline
        $\varepsilon_{3}$  & 8,33  & $\varepsilon_{13}$ & 1,25   & $\varepsilon_{23}$ & 6,50  \\ \hline
        $\varepsilon_{4}$  & 9,10  & $\varepsilon_{14}$ & -1,75  & $\varepsilon_{24}$ & -4,07 \\ \hline
        $\varepsilon_{5}$  & 1,06  & $\varepsilon_{15}$ & -13,33 & $\varepsilon_{25}$ & -3,66 \\ \hline
        $\varepsilon_{6}$  & 6,84  & $\varepsilon_{16}$ & -39,94 & $\varepsilon_{26}$ & 16,26 \\ \hline
        $\varepsilon_{7}$  & -0,91 & $\varepsilon_{17}$ & 1,34   & $\varepsilon_{27}$ & 8,80  \\ \hline
        $\varepsilon_{8}$  & 3,33  & $\varepsilon_{18}$ & -1,66  & $\varepsilon_{28}$ & 17,78 \\ \hline
        $\varepsilon_{9}$  & -7,43 & $\varepsilon_{19}$ & 12,06  & $\varepsilon_{29}$ & 18,43 \\ \hline
        $\varepsilon_{10}$ & -0,43 & $\varepsilon_{20}$ & -20,75 & $\varepsilon_{30}$ & 19,06 \\ \hline
    \end{tabular}
\end{table}
\begin{align*}
    R^2 = 1 - \frac{
        \tfrac{1}{n} \sum\limits_{i=1}^n \varepsilon_i^2
    }{\Db(y)} = 146,68
\end{align*}

\end{document}