\documentclass[utf8, a4paper, 14pt, russian, oneside]{book}

% Кодировка
\usepackage[T2A]{fontenc}
\usepackage[utf8]{inputenc}
\usepackage[main=russian, english]{babel}

% Пакеты для работы с математикой
\usepackage{amsmath}
\usepackage{amsfonts}
\usepackage{amssymb}

% Вставка изображений
\usepackage{graphicx}

% Пакет для работы с таблицами
\usepackage{tabularx}
\usepackage{booktabs}
\usepackage{longtable}

% Для больших множеств
\usepackage{mathtools}

% Для работы с рисунками
\usepackage{caption}

% Для создания графов в 3 блоке
\usepackage[all]{xy}

% Для специальных символов
\usepackage{textcomp}
\newcommand{\mysec}[1]{
{\center\section*{#1}}
\addcontentsline{toc}{section}{#1}
}

% Команды для настройки содержания
\renewcommand\contentsname{\center{Содержание}} % Вместо оглавления пишется содержание
\addto{\captionsenglish}{\renewcommand{\bibname}{References}}
\begin{document}

\thispagestyle{empty}
~\vspace{-2cm}\setlength{\parindent}{0cm}
\begin{center}
	\includegraphics[scale=1.5]{../include/logo.png}\\[2pt]
	МИНОБРНАУКИ РОССИИ\\
	Федеральное государственное бюджетное образовательное учреждение\\
	высшего профессионального образования\\[5pt]
	\textbf{<<МИРЭА – Российский технологический университет>>}\\[5pt]
	\textbf{\large РТУ МИРЭА}\\[20pt]
	\hrule{}\mbox{}\\[1pt]
	\hrule{}\mbox{}\\[20pt]	
	Институт кибернетики \\ Кафедра <<Инфориационная безопасность>> (БК №252)\\[35pt]
	\textbf{Долгосрочное задание} \\
	по дисциплине: Математическая статистика
\end{center}
	\vspace{4in}
	Студент группы ККСО-01-19:  \qquad \qquad \qquad  \qquad     Колесников А.В.
\vspace{0.6in}
\begin{center}
Москва --- 2021
\end{center}
\newpage

\tableofcontents
\newpage

\mysec{Описание данных}

В качестве данных возьмем данные о цене на курс криптовалюты <<Ethereum>> (в долларах) в период времени с 8 ноября по 7 декабря. Объём выборки равен 30.
Пусть X - цена криптовалюты.
\begin{table}[h!]
    \centering
    \begin{tabular}{|c|c|c|}
        \hline
        № & Дата & Цена \\ \hline
        1 & 11.08 & 4619,65 \\ \hline
        2 & 11.09 & 4810,07 \\ \hline
        3 & 11.10 & 4733,36 \\ \hline
        4 & 11.11 & 4635,45 \\ \hline
        5 & 11.12 & 4724,31 \\ \hline
        6 & 11.13 & 4666,72 \\ \hline
        7 & 11.14 & 4648,63 \\ \hline
        8 & 11.15 & 4627,09 \\ \hline
        9 & 11.16 & 4570,48 \\ \hline
        10 & 11.17 & 4213,91 \\ \hline
        11 & 11.18 & 4287,80 \\ \hline
        12 & 11.19 & 3995,73 \\ \hline
        13 & 11.20 & 4298,35 \\ \hline
        14 & 11.21 & 4412,20 \\ \hline
        15 & 11.22 & 4266,51 \\ \hline
        16 & 11.23 & 4089,68 \\ \hline
        17 & 11.24 & 4340,04 \\ \hline
        18 & 11.25 & 4271,39 \\ \hline
        19 & 11.26 & 4522,21 \\ \hline
        20 & 11.27 & 4043,00 \\ \hline
        21 & 11.28 & 4101,65 \\ \hline
        22 & 11.29 & 4296,95 \\ \hline
        23 & 11.30 & 4447,77 \\ \hline
        24 & 12.01 & 4623.68 \\ \hline
        25 & 12.02 & 4586,33 \\ \hline
        26 & 12.03 & 4514,36 \\ \hline
        27 & 12.04 & 4227,76 \\ \hline
        28 & 12.05 & 4119,63 \\ \hline
        29 & 12.06 & 4199,00 \\ \hline
        30 & 12.07 & 4369,08 \\ \hline
    \end{tabular}
\end{table}
\newpage

\mysec{График динамики ряда}
Временной ряд - это собранный в разные моменты времени статистический материал о значении каких-либо параметров.

\begin{figure}[h!]
    \centering
    \includegraphics{img/dynamic_range.png}
    \caption{Динамика ряда.}
\end{figure}

\mysec{Автокорреляционная функция}
Автокорреляция - это обычный коэффициент корреляции для значений одной величины в разные моменты времени (лаги).
Серию коэффициентов автокорреляции уровней ряда с последовательным увеличением величины лага принято называть автокорреляционной функцией.
Автокорреляция в момент вермени $k$ вычисляется по следующей формуле:
\begin{gather*}
    r(k) = \frac{\cov(X, Y)}{\sqrt{\Db(X)}\sqrt{\Db(Y)}}
    = \frac{
        \overline{x_k \cdot x_{k-1}} - \overline{x_k} \cdot \overline{x_{k-1}}
    }{
        \sigma_k \cdot \sigma_{k-1}
    }
\end{gather*}

Если $X = \{ x_1, \ldots, x_{30} \}$, то $X_k = \{ x_1, \ldots, x_{n-k} \}, X_{k-1} = \{ x_{1+k}, \ldots, x_n \}, k \in \overline{1, \tfrac{n}{2}}, m = n - k$.

Найдём $r(1)$. Для этого необходимо вычислить вспомогательные значения:
\begin{gather*}
    X_1 = \{x_1, \ldots, x_{29}\} \qquad X_2 = \{x_2, \ldots, x_{30}\} \\
    \overline{x_k} = \frac{1}{29} \sum_{i=1}^{29} x_i = \frac{4619,65 + \ldots + 4199,00}{29} = 4410,13 \\ 
    \overline{x_{k-1}} = \sum_{i=2}^{30} x_i = \frac{4810,07 + \ldots + 4369,08}{29} = 4401,49 \\
    D_k = \frac{1}{29} \sum_{i=1}^{29} x_i^2 - (\overline{x_k})^2 = \frac{4619,65^2 + \ldots + 4199,00^2}{29} - 4410,13^2 = 53255,96 \\
    D_{k-1} = \frac{1}{29} \sum_{i=2}^{30} x_i^2 - (\overline{x_{k-1}})^2 = \frac{4810,07^2 + \ldots + 4369,08^2}{29} - 4401,49^2 = 51722,63 \\
    \sigma_k = \sqrt{D_k} = 230,77 \\
    \sigma_{k-1} = \sqrt{D_{k-1}} = 227,43 \\
    \overline{x_k \cdot x_{k-1}} = \frac{4619,65 \cdot 4810,07 + \ldots + 4199,00 \cdot 4369,08}{49} = 19445863,97 \\
    r(1) = \frac{19445863,97 - 4410,13 \cdot 4401,49}{230,77 \cdot 227,43} = 0,6617
\end{gather*}
Аналогично найдем $r(2), \ldots, r(15)$. Приведём итоговый результат в виде таблицы, опустив промежуточные вычисления:
\begin{table}[h!]
    \centering    
    \begin{tabular}{|c|c|}
        \hline
        $k$ & $r(k)$ \\ \hline
        1 & 0,6617 \\ \hline
        2 & 0,4424 \\ \hline
        3 & 0,1949 \\ \hline
        4 & 0,1888 \\ \hline
        5 & 0,1939 \\ \hline
        6 & 0,1814 \\ \hline
        7 & -0,0254 \\ \hline
        8 & -0,0599 \\ \hline
        9 & -0,2435 \\ \hline
        10 & -0,2142 \\ \hline
        11 & -0,1569 \\ \hline
        12 & -0,2445 \\ \hline
        13 & -0,4004 \\ \hline
        14 & -0,4425 \\ \hline
        15 & -0,1600 \\ \hline
    \end{tabular}
\end{table}

\begin{figure}[h!]
    \centering
    \includegraphics{img/correlogramma.png}
    \caption{Коррелограмма}
\end{figure}
Поскольку коррелограмма АКФ имеет максимум при $k=1 (r(1) = 0,6617)$, то ряд содержит только тенденцию (тренд).
Тренд - это долговременная тенденция изменения исследуемого временного ряда.

\newpage

\mysec{Уравнение тренда}
На основе коррелограммы опишем тренд линейной функцией:
\begin{gather*}
    x(t) = kt + b
\end{gather*}
Для вычисления коэффициентов $k$ и $b$ воспользуемся функцией $\varPhi(k, b) = \sum\limits_{i=1}^n (x_i - kt_i -b)^2$. Для нахождения коэффициентов $k$ и $b$ решим следующую систему уравнений:
\begin{gather*}
    \begin{dcases}
        \der{\varPhi}{k} = 0, \\
        \der{\varPhi}{b} = 0
    \end{dcases}
    \Rightarrow
    \begin{dcases}
        \der{\varPhi}{k} = -2 \sum\limits_{i=1}^n (x_it_i - t_i^2k - t_ib) = 0, \\
        \der{\varPhi}{b} = -2 \sum\limits_{i=1}^n (x_i - kt_i - b) = 0
    \end{dcases}
    \Rightarrow \\
    \begin{dcases}
        k \sum\limits_{i=1}^n t_i^2 + b \sum\limits_{i=1}^n t_i = \sum\limits_{i=1}^n x_it_i, \\
        k \sum\limits_{i=1}^n t_i + nb = \sum\limits_{i=1}^n x_i
    \end{dcases}
    \Rightarrow \\
    \begin{dcases}
        9455k + 465b = 2019386,97, \\
        465k + 30 b = 132262,79 \\
    \end{dcases}
    \Rightarrow \\
    \begin{dcases}
        k\left(9455 - \frac{465^2}{30}\right) = 2019386,97 - \frac{465 \cdot 132262,79}{30}, \\
        b = \frac{132262,79 - 465k}{30}
    \end{dcases}
    \Rightarrow
    \begin{dcases}
        k = -13,65, \\
        b = 4620,39
    \end{dcases}
\end{gather*}
Следовательно, уравнение тренда будет выглядеть следующим образом:
\begin{gather*}
    x(t) = -13,65t + 4620,39
\end{gather*}
\newpage

\mysec{Ошибки тренда}
Определим ошибки тренда $\xi_i$ по следующей формуле:
\begin{gather*}
    x_i = -13,65 t_i + 4620,39 + \xi_i \Rightarrow \xi_i = x_i + 13,65 t_i - 4620,39 \\
    \xi_1 = 4619,65 + 13,65 \cdot 1 - 4620,39 = 12,9144 \\
    \ldots \\
    \xi_{30} = 4369,08 + 13,65 \cdot 30 - 4620,39 = 158,2963
\end{gather*}

Получим следующую таблицу ошибок:
\begin{table}[h!]
    \centering
    \begin{tabular}{|c|c|c|c|}
        \hline
        $i$ & $\xi_i$ & $i$ & $\xi_i$ \\ \hline
        1 & 12,9144    & 16 & -312,2529 \\ \hline
        2 & 216,9879   & 17 &  -48,2394 \\ \hline
        3 & 153,9314   & 18 & -103,2359 \\ \hline
        4 & 69,6749    & 19 &  161,2376 \\ \hline
        5 & 172,1884   & 20 & -304,3188 \\ \hline
        6 & 128,2519   & 21 & -232,0153 \\ \hline
        7 & 123,8155   & 22 &  -23,0618 \\ \hline
        8 & 115,9290   & 23 &  141,4117 \\ \hline
        9 & 72,9725    & 24 &  330,9752 \\ \hline
        10 & -269,9440 & 25 &  307,2787 \\ \hline
        11 & -182,4005 & 26 &  248,9622 \\ \hline
        12 & -460,8170 & 27 &  -23,9842 \\ \hline
        13 & -144,5435 & 28 & -118,4607 \\ \hline
        14 & -17,0399  & 29 &  -25,4372 \\ \hline
        15 & -149,0764 & 30 &  158,2963 \\ \hline
    \end{tabular} 
\end{table}

\newpage
По этой таблице построим следующую диаграмму ошибок:

\begin{figure}[h!]
    \centering
    \includegraphics{img/errors.png}
    \caption{Диаграмма ошибок.}
\end{figure}

Определим коэффициент детерминации для ошибок $\xi$:
\begin{gather*}
    R^2 = 1 - \frac{\dfrac{1}{n}\sum\limits_{i=1}^n \xi_i^2}{\Db(X)} \\
    \overline{\xb} = \frac{1}{n} \sum\limits_{i=1}^n x_i = \frac{4619,65 + \ldots + 4369,08}{30} = 4408,76 \\
    \Db(X) = \frac{1}{30} \sum\limits_{i=1}^n x_i^2 - (\overline{\xb})^2 = 53330,38
\end{gather*}

Средняя ошибка:
\begin{gather*}
    \frac{1}{n} \sum_{i=1}^n \xi_i^2 = 37586,85
\end{gather*}

Коэффициент детерминации для ошибок $\xi$:
\begin{gather*}
    R^2 = 1 - \frac{37586,85}{53330,38} = 0,2952
\end{gather*}

\newpage

\mysec{Коэффициент автокорреляции ошибок с лагом 1}

Коэффициент корреляции $r_{k, k-1}(1)$ находится по следующей формуле
\begin{gather*}
    r_\xi(k) = \frac{
        \overline{\xi_k \cdot \xi_{k-1}} - \overline{\xi_k} \cdot \overline{\xi_{k-1}}
    }{
        \sigma_k \cdot \sigma_{k-1}
    }
\end{gather*}
и будет находиться между группами  $\xi_k$ и $\xi_{k-1}$:
\begin{table}[h!]
    \centering
    \begin{tabular}{|c|c|}
        \hline
        $\xi_k$ & $\xi_{k-1}$ \\ \hline
        12,9144   & 216,9879  \\ \hline
        216,9879  & 153,9314  \\ \hline
        153,9314  & 69,6749   \\ \hline
        69,6749   & 172,1884  \\ \hline
        172,1884  & 128,2519  \\ \hline
        128,2519  & 123,8155  \\ \hline
        123,8155  & 115,9290  \\ \hline
        115,9290  & 72,9725   \\ \hline
        72,9725   & -269,9440 \\ \hline
        -269,9440 & -182,4005 \\ \hline
        -182,4005 & -460,8170 \\ \hline
        -460,8170 & -144,5435 \\ \hline
        -144,5435 & -17,0399  \\ \hline
        -17,0399  & -149,0764 \\ \hline
        -149,0764 & -312,2529 \\ \hline
        -312,2529 & -48,2394  \\ \hline
        -48,2394  & -103,2359 \\ \hline
        -103,2359 & 161,2376  \\ \hline
        161,2376  & -304,3188 \\ \hline
        -304,3188 & -232,0153 \\ \hline
        -232,0153 & -23,0618  \\ \hline
        -23,0618  & 141,4117  \\ \hline
        141,4117  & 330,9752  \\ \hline
        330,9752  & 307,2787  \\ \hline
        307,2787  & 248,9622  \\ \hline
        248,9622  & -23,9842  \\ \hline
        -23,9842  & -118,4607 \\ \hline
        -118,4607 & -25,4372  \\ \hline
        -25,4372  & 158,2963  \\ \hline
    \end{tabular}
\end{table}

Получим, что $r_{k, k-1}(1) = 0,5377$.

\mysec{Оценка коэффициентов уравнений остатка}

Рассмотрим $\xi_i = a_1 \xi_{i-1} + a_0 + \varepsilon_i, i \in \overline{2, 30}$. Найдем коэффициенты $a_1$ и $a_0$ уравнения:
\begin{gather*}
    \xi_i = a_1 \xi_{i-1} + a_0
\end{gather*}
Оценим при помощи метода наименьших квадратов $\varPhi(a_1, a_0) = \sum\limits_{i=1}^n (\xi_i - a_1\xi_{i-1} - a_0)^2, n = 30$.
\begin{gather*}
    \begin{dcases}
        \der{\varPhi}{a_1} = -2 \sum\limits_{i=2}^n(\xi_i \xi_{i-1} - \xi_{i-1}^2 a_1 - \xi_{i-1}a_0) = 0, \\
        \der{\varphi}{a_0} = -2 \sum\limits_{i=2}^n (\xi_i - a_1\xi_{i-1} - a_0) = 0
    \end{dcases}
    \Rightarrow \\
    \begin{dcases}
        a_1 \sum\limits_{i=2}^n\xi_{i-1}^2 + a_0\sum\limits_{i=2}^n\xi_{i-1} = \sum\limits_{i=2}^n \xi_i \xi_{i-1}, \\
        a_1 \sum\limits_{i=2}^n \xi_{i-1} + (n-1)a_0 = \sum\limits_{i=2}^n\xi_i
    \end{dcases}
    \Rightarrow \\
    \begin{dcases}
        1127438,6227 a_1 -12,9144 a_0 = 599344,7436, \\
        -12,9144a_1 + 29a_0 = -158,2963
    \end{dcases}
    \Rightarrow
    \begin{dcases}
        a_1 = 0,5315, \\
        a_0 = -5,2218
    \end{dcases}
    \Rightarrow \\
    \xi_i = 0,5315 \xi_{i-1} -5,2218
\end{gather*}

Вычислим ошибки $\varepsilon_i$:
\begin{gather*}
    \varepsilon_i = \xi_i - 0,5315\xi_{i-1} + 5,2218 \\
    \varepsilon_2 = \xi_2 - 0,5315\xi_{1} + 5,2218 = -97,2013
\end{gather*}

\newpage
По аналогии посчитаем остальные ошибки:
\begin{table}[h!]
    \centering
    \begin{tabular}{|c|c|c|c|}
        \hline
        $i$ & $\varepsilon_i$ & $i$ & $\varepsilon_i$ \\ \hline
           &           & 16 & 22,1199   \\ \hline
        2  & -97,2013  & 17 & -281,3900 \\ \hline
        3  & 140,3892  & 18 & 11,8563   \\ \hline
        4  & 122,1183  & 19 & -183,7181 \\ \hline
        5  & -16,6281  & 20 & 328,2167  \\ \hline
        6  & 109,2393  & 21 & -175,7719 \\ \hline
        7  & 67,6610   & 22 & -214,5353 \\ \hline
        8  & 67,4165   & 23 & -93,0058  \\ \hline
        9  & 82,3631   & 24 & -29,2927  \\ \hline
        10 & 221,6800  & 25 & 172,8665  \\ \hline
        11 & -167,7693 & 26 & 180,1674  \\ \hline
        12 & 67,7634   & 27 & 266,9326  \\ \hline
        13 & -378,7647 & 28 & 44,2040   \\ \hline
        14 & -130,2643 & 29 & -99,7181  \\ \hline
        15 & 67,4217   & 30 & -104,3560 \\ \hline
    \end{tabular}
\end{table}

Вычислим коэффициент детерминации для ошибок $\varepsilon$:
\begin{gather*}
    R^2 = 1 - \frac{
        \dfrac{1}{n-1} \sum\limits_{i=1}^{n-1} \varepsilon_i^2 + \xi_1^2
    }{
        \Db(X)
    } \\ 
    R^2 = 1 - \frac{
        \dfrac{1}{30 - 1} (-97,2013^2 + \ldots + -104,3560^2 + 12,9144^2)
    }{
        53330,3766
    } = 0,4935.
\end{gather*}

Для процесса $\xi_i = a_1 \xi_{i-1} + a_0 + \varepsilon_i$ условие стационарности означает, что $|a_1| < 1$, поскольку из $1 - a_1z = 0 \Rightarrow |z| = \tfrac{1}{a_1} > 1$. Тогда
\begin{gather*}
    z = \frac{1}{0,5315} = 1,88 > 1 \Rightarrow\  \text{процесс } \xi_1, \ldots, \xi_{30} \  \text{стационарен.}
\end{gather*}

\mysec{Построение авторегрессионного процесса 2-го порядка. Коэффициент детерминации}

Авторегрессионный процесс 2-го порядка:
\begin{gather*}
    Y_i = a_0 + a_1 Y_{i-1} + a_2 Y_{i-2} + \varepsilon_i, i \in \overline{3, 30}.
\end{gather*}
Оценим при помощи метода наименьших квадратов:
\begin{gather*}
    \varPhi(a_2, a_1, a_0) = \sum\limits_{i=3}^n (Y_i - a_0 - a_1Y_{i-1} - a_2Y_{i-2})^2 \\ 
    \begin{dcases}
        \der{\varPhi}{a_2} = -2 \sum\limits_{i=3}^n (Y_i - a_2Y_{i-2} - a_1Y_{i-1} - a_0)Y_{i-3} = 0, \\
        \der{\varPhi}{a_1} = -2 \sum\limits_{i=3}^n (Y_i - a_2Y_{i-2} - a_1Y_{i-1} - a_0)Y_{i-2} = 0, \\
        \der{\varPhi}{a_0} = -2 \sum\limits_{i=3}^n (Y_i - a_2Y_{i-2} - a_1Y_{i-1} - a_0) = 0
    \end{dcases}
    \Rightarrow \\
    \begin{dcases}
        -\sum\limits_{i=3}^nY_iY_{i-2} + a_2 \sum\limits_{i=3}^nY^2_{i-2} + a_1\sum\limits_{i=3}^nY_{i-1}Y_{i-2} + a_0\sum\limits_{i=3}^nY_{i-2} = 0, \\
        -\sum\limits_{i=3}^nY_iY_{i-1} + a_2 \sum\limits_{i=3}^nY_{i-2}Y_{i-1} + a_1 \sum\limits_{i=3}^nY^2_{i-1} + a_0\sum\limits_{i=3}^nY_{i-1} = 0, \\
        -\sum\limits_{i=3}^nY_i + a_2 \sum\limits_{i=3}^nY_{i-2} + a_1 \sum\limits_{i=3}^nY_{i-1} + (n-2) \cdot  a_0 = 0
    \end{dcases}
    \Rightarrow \\
    \begin{dcases}
        a_2\sum\limits_{i=3}^n Y_{i-2}^2 + a_1 \sum\limits_{i=3}^n Y_{i-1}Y_{i-2} + a_0\sum\limits_{i=3}^n Y_{i-2} = \sum\limits_{i=3}^n Y_iY_{i-2}, \\
        a_2\sum\limits_{i=3}^nY_{i-2}Y_{i-1} + a_1 \sum\limits_{i=3}^nY^2_{i-1} + a_0\sum\limits_{i=3}^nY_{i-1} = \sum\limits_{i=3}^nY_iY_{i-1}, \\
        a_2\sum\limits_{i=3}^nY_{i-2} + a_1\sum\limits_{i=3}^nY_{i-1} + (n-2)\cdot a_0 = \sum\limits_{i=3}^nY_i
    \end{dcases}
    \Rightarrow \\
    \begin{dcases}
        540183495,3 a_2 + 541709215,4 a_1 + 122833,07 a_0 = 543259744,9, \\
        541709215,4 a_2 + 563320268,8 a_1 + 127643,14 a_0 = 563930055,3, \\
        122833,07 a_2 + 127643,14 a_1 + 28 a_0 = 122833,07
    \end{dcases}
    \Rightarrow \\
    \begin{dcases}
        a_2 = 0,0259, \\
        a_1 = -0,2394, \\
        a_0 = 5364,93
    \end{dcases}
\end{gather*}

Следовательно, получим следующее уравнение авторегрессионного процесса 2-го порядка:
\begin{gather*}
    Y_i = 5364,9308-0,2395\cdot Y_i-1 0,0259\cdot Y_{i-2}
\end{gather*}

Найдем ошибки модели по следующей формуле:
\begin{gather*}
    \varepsilon_i = Y_i - a_0 - a_1Y_{i-1} - a_2Y_{i-2} = Y_i - 5364,9308 + 0,2395\cdot Y_{i-1} + 0,0259\cdot Y_{i-2}
\end{gather*}

Вычисленные значения:
\begin{table}[h!]
    \centering
    \begin{tabular}{|c|c|c|c|}
        \hline
        $i$ & $\varepsilon_i$ & $i$ & $\varepsilon_i$ \\ \hline
           &           & 16 & -36,9637  \\ \hline
           &           & 17 & -231,4777 \\ \hline
        3  & 283,9814  & 18 & -346,5806 \\ \hline
        4  & 458,5674  & 19 & -119,1522 \\ \hline
        5  & 356,1120  & 20 & -115,3366 \\ \hline
        6  & 280,9708  & 21 & 19,2158   \\ \hline
        7  & 356,5089  & 22 & -451,0060 \\ \hline
        8  & 295,1448  & 23 & -349,4947 \\ \hline
        9  & 273,3624  & 24 & -122,6339 \\ \hline
        10 & 247,4975  & 25 & 71,2755   \\ \hline
        11 & 103,5921  & 26 & 240,1050  \\ \hline
        12 & -227,7236 & 27 & 192,9407  \\ \hline
        13 & -231,6053 & 28 & 55,1421   \\ \hline
        14 & -454,1587 & 29 & -259,4049 \\ \hline
        15 & -120,5052 & 30 & -352,9323 \\ \hline
    \end{tabular} 
\end{table}

Средняя ошибка:
\begin{gather*}
    \frac{1}{n-2}\sum_{i=3}^n\varepsilon_i^2 = \frac{1}{28} \sum_{i=3}^{30} = 72223,92058 
\end{gather*}

Коэффициент детерминации:
\begin{gather*}
    R^2_{\varepsilon} = 1 - \frac{\dfrac{1}{n-2} \sum\limits_{i=3}^n \varepsilon_i^2}{\Db(X)} = 1 - \frac{72223,9206}{53330,3766} = -0,3543
\end{gather*}

\newpage

Определим с помощью уравнения $Y_i = -0,0259\cdot Y_{i-2} - 0,2395 \cdot Y_{i-1} + 5364,9308$ теоретические значения $Y_I, i \in \overline{31,38}$.
\begin{table}[h!]
    \centering
    \begin{tabular}{|c|c|c|}
        \hline
        $k$ & $Y_i$ теоретическое & $Y_i$ реальное \\  \hline
        1  & 4209,78 & 4210,21 \\ \hline
        2  & 4243,62 & 4240,67 \\ \hline
        3  & 4239,57 & 4220,39 \\ \hline
        4  & 4239,73 & 4250,97 \\ \hline
        5  & 4239,70 & 4240,48 \\ \hline
        6  & 4239,71 & 4270,15 \\ \hline
        7  & 4239,71 & 4269,12 \\ \hline
        8  & 4239,71 & 4260,78 \\ \hline
    \end{tabular}
\end{table}

При увеличении $k$ теоретическое значение будет все сильнее отличаться от реальных данных.

\mysec{Заключение}
Временной ряд содержит тенденцию.
Т.к. значения коэффициентов детерминации $R_1^2$ и $R_2^2$ находятся на достаточном расстоянии по модулю от 1, то линейное уравнение регрессии
не может достаточно точно описать данный временной ряд. Т.е. цена криптовалюты зависит не от дня, а от более сложных показателей, например новости, общее состояние рынка и прочее.

Средние ошибки обоих моделей достаточно велики, что говорит о более сложном устройстве данной выборки.

\end{document}