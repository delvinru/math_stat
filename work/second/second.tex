\documentclass[utf8, a4paper, 14pt, russian, oneside]{book}

% Кодировка
\usepackage[T2A]{fontenc}
\usepackage[utf8]{inputenc}
\usepackage[main=russian, english]{babel}

% Пакеты для работы с математикой
\usepackage{amsmath}
\usepackage{amsfonts}
\usepackage{amssymb}

% Вставка изображений
\usepackage{graphicx}

% Пакет для работы с таблицами
\usepackage{tabularx}
\usepackage{booktabs}
\usepackage{longtable}

% Для больших множеств
\usepackage{mathtools}

% Для работы с рисунками
\usepackage{caption}

% Для создания графов в 3 блоке
\usepackage[all]{xy}

% Для специальных символов
\usepackage{textcomp}
\newcommand{\mysec}[1]{
{\center\section*{#1}}
\addcontentsline{toc}{section}{#1}
}

% Команды для настройки содержания
\renewcommand\contentsname{\center{Содержание}} % Вместо оглавления пишется содержание
\addto{\captionsenglish}{\renewcommand{\bibname}{References}}
\begin{document}

\thispagestyle{empty}
~\vspace{-2cm}\setlength{\parindent}{0cm}
\begin{center}
	\includegraphics[scale=1.5]{../include/logo.png}\\[2pt]
	МИНОБРНАУКИ РОССИИ\\
	Федеральное государственное бюджетное образовательное учреждение\\
	высшего профессионального образования\\[5pt]
	\textbf{<<МИРЭА – Российский технологический университет>>}\\[5pt]
	\textbf{\large РТУ МИРЭА}\\[20pt]
	\hrule{}\mbox{}\\[1pt]
	\hrule{}\mbox{}\\[20pt]	
	Институт кибернетики \\ Кафедра <<Инфориационная безопасность>> (БК №252)\\[35pt]
	\textbf{Долгосрочное задание} \\
	по дисциплине: Математическая статистика
\end{center}
	\vspace{4in}
	Студент группы ККСО-01-19:  \qquad \qquad \qquad  \qquad     Колесников А.В.
\vspace{0.6in}
\begin{center}
Москва --- 2021
\end{center}
\newpage

\tableofcontents
\newpage

\mysec{Описание данных и обоснование разделения на <<малые>> и <<большие>> выборки.}

Для изучения возьмём данные переписи населения в Швейцарии в разные периоды времени (в тыс. человек):
\begin{table}[h!]
    \centering
    \small
    \begin{tabular}{|c|c|c|}
        \hline
        & \textbf{Мужчины} & \textbf{Женщины} \\ \hline
        \multicolumn{3}{|c|}{0-15} \\ \hline
        1980 & 638 & 608 \\ \hline
        1990 & 585 & 559 \\ \hline
        2000 & 642 & 606 \\ \hline
        2002 & 627 & 595 \\ \hline
        2005 & 616 & 583 \\ \hline

        2007 & 606 & 573 \\ \hline
        2009 & 608 & 673 \\ \hline
        2011 & 612 & 589 \\ \hline
        2015 & 631 & 598 \\ \hline
        2019 & 661 & 626 \\ \hline

        \multicolumn{3}{|c|}{16-59} \\ \hline
        1980 & 1951 & 1965 \\ \hline
        1990 & 2148 & 2114 \\ \hline
        2000 & 2250 & 2230\\ \hline
        2002 & 2290 & 2268 \\ \hline
        2005 & 2329 & 2309 \\ \hline

        2007 & 2362 & 2337 \\ \hline
        2009 & 2424 & 2289 \\ \hline
        2011 & 2480 & 2432 \\ \hline
        2015 & 2581 & 2521 \\ \hline
        2019 & 2425 & 2557 \\ \hline

        \multicolumn{3}{|c|}{Больше 60} \\ \hline
        1980 & 482 & 670 \\ \hline
        1990 & 540 & 760 \\ \hline
        2000 & 613 & 828 \\ \hline
        2002 & 641 & 859 \\ \hline
        2005 & 694 & 901 \\ \hline

        2007 & 734 & 936 \\ \hline
        2009 & 774 & 850 \\ \hline
        2011 & 804 & 997 \\ \hline
        2015 & 880 & 1063 \\ \hline
        2019 & 963 & 1135 \\ \hline
    \end{tabular}
\end{table}

Разделим данные на 3 "малые" выборки по возрастной группе: люди от 0 до 15, от 16 до 59 и старше 60 лет.
Поделим данные на 2 "большие" выборки по половому признаку: мужчины и женщины.
\newpage

\mysec{Проверка гипотезы о равенстве средних всех <<малых>> выборок.}

Прежде чем проверять равенство средних, нужно проверить равенство дисперсий.
В предположении нормальности распределения проверим следующие гипотезы:
\begin{enumerate}
    \item $H_0: \overline{\Db(x)} = \overline{\Db(y)}$
    \item $H_0: \overline{\Db(x)} = \overline{\Db(z)}$
    \item $H_0: \overline{\Db(y)} = \overline{\Db(z)}$
\end{enumerate} 

1) Найдем исправленное выборочное дисперсий для первой и второй выборки по формуле:
\begin{gather*}
    S^2 = \frac{n}{n-1} \overline{\Db(x)} = \frac{\sum\limits_{i=1}^{v}(x_i - \overline{\xb})}{n-1} \\
    \overline{\xb} = \frac{638 + 608 + 585 + \ldots + 598 + 661 + 626}{20} = 611,8 \\
    \overline{\yb} = \frac{1951 + 1965 + 2148 + \ldots + 2521 + 2425 + 2557}{20}= 2313,1 \\
    S^2(x) = \frac{10}{9} \cdot \left((638 - 611,8)^2 + \ldots + (626 - 611,8)^2\right) = 17236,89 \\ 
    S^2(y) = \frac{10}{9} \cdot \left( (1951 - 2313,1)^2 + \ldots + (2557 - 2313,1)^2  \right) = 641944,22 \\
    F_\text{набл} = \frac{S^2(y)}{S^2(x)} = 37,24
\end{gather*}
Распределение Фишера со степенями свободы $df = 10 - 1 = 9$ и $\alpha = 0,05$ равно $F_\text{табл} = 3,18$.
Поскольку $F_\text{набл} > F_\text{табл} \Rightarrow$ гипотеза о равенстве дисперсий $H_0$ отклоняется.
\newpage

В таком случае проверим гипотезу о равенстве математических ожиданий $M(x) = M(y)$.
\begin{gather*}
    t = \frac{\overline{\xb} - \overline{\yb}}{\sqrt{\frac{S^2(x)}{n_x} +\frac{S^2(y)}{n_y}}}\\
    t_\text{набл} = \frac{611,8 - 2313,1}{
        \sqrt{
            \frac{17236,89}{10} + \frac{641944,22}{10}
        }
    } = -6.626\\
    df = \frac{
        \left(
            \frac{S^2(x)}{n_x} + \frac{S^2(y)}{n_y}
        \right)^2
    }{
        \frac{
            \frac{S^2(x)}{n_x}
        }{
            n_x + 1
        }
        +
        \frac{
            \frac{S^2(y)}{n_y}
        }{
            n_y + 1
        }
    }
    - 2 = 5990.55\\
    t_{\text{табл}} = 2.16
\end{gather*}
Поскольку, $|t_{\text{набл}}| > t_{\text{табл}} \Rightarrow$ гипотеза о равенстве математических ожиданий отвергается.

2) Найдем исправленное выборочное дисперсий для первойй и третей выборки:
\begin{gather*}
    S^2(x) = 17236,89\\
    \overline{\zb} = \frac{482 + 670 + 540 + \ldots + 1063 + 963 + 1135}{20} = 806,2 \\
    S^2(z) = \frac{10}{9} \cdot \left( (482 - 806,2)^2 + \ldots + (1135 - 806,2)^2 \right) = 612914,67\\
    F_{\text{набл}} = \frac{S^2(z)}{S^2(x)} = 35,56
\end{gather*}
Распределение Фишера со степенями свободы $df = 9$ и $\alpha = 0,05$ равно $F_{\text{табл}} = 3,18$. Поскольку $F_{\text{набл}} > F_{\text{табл}} \Rightarrow$ гипотеза о равенстве дисперсий $H_0$ отклоняется.

В таком случае проверим гипотезу о равенстве математических ожиданий $M(x) = M(z)$.
\begin{gather*}
    t = \frac{
        \overline{\xb} - \overline{\yb}
    }{
        \sqrt{\frac{S^2(x)}{n_x} +\frac{S^2(y)}{n_y}}
    } \\
    t_{\text{набл}} = \frac{611,8 - 806,2}{
        \sqrt{
            \frac{17236,89}{10} + \frac{612914,67}{10}
        }
    } = -0,77\\
    df = 5726,65 \\
    t_{\text{табл}} = 2.16
\end{gather*}
Поскольку, $|t_{\text{набл}}| < t_{\text{табл}} \Rightarrow$ гипотеза о равенстве математических ожиданий принимается.

3) Найдем исправленное выборочное дисперсий для второй и третей выборки:
\begin{gather*}
    S^2(y) = 641944,22 \\
    S^2(z) = 612914,67 \\
    F_{\text{набл}} = \frac{S^2(y)}{S^2(z)} = 1,05
\end{gather*}
Распределение Фишера со степенями свободы $df = 9$ и $\alpha = 0,05$ равно $F_{\text{табл}} = 3,18$.

Поскольку, $F_{\text{набл}} < F_{\text{табл}} \Rightarrow$ гипотеза о равенстве дисперсий $H_0$ принимается.

Таким образом, статистика критерия находится следующим образом:
\begin{gather*}
    t = \frac{
        \overline{\yb} - \overline{\zb}
    }{
        \sqrt{(n_yS^2(y) + n_zS^2(z))}
    }
    \cdot
    \sqrt{
        \frac{n_yn_z(n_y + n_z - 2)}{n_y+n_z}
    } \\
    t_{\text{набл}} = 4,03\\
    t_{\text{табл}} = (df_{n_y} + df_{n_z} - 2) = t_{\text{табл}}(18) = 2,1
\end{gather*}
Таким образом, $|t_{\text{набл}}| > t_{\text{табл}} \Rightarrow$ гипотеза отвергается.

\newpage
\mysec{Проврека гипотезы о равенстве средних <<больших>> выборок}

Найдем $\overline{\xb}$ для каждой <<большой>> выборки.
\begin{gather*}
    \overline{\xb} = \frac{3071 + 3243 + 3273 + 3433 + 3505 + 3664 + 3558 + 3722 + 3639 + 3793}{10} = \\
    = 3490,1 \\
    \overline{\yb} = \frac{3072 + 3846 + 3806 + 3812 + 3896 + 4018 + 4092 + 4182 + 4049 + 4318}{10} = \\
    = 3972,1
\end{gather*}
Найдем исправленное выборочное двух <<больших>> выборок:
\begin{gather*}
    S^2(x) = \frac{10}{9} \cdot \left( (3071 - 3490,1)^2 + \ldots + (3793 - 3490,1)^2 \right) = 544296,56\\
    S^2(y) = \frac{10}{9} \cdot \left( (3072 - 3972,1)^2 + \ldots + (4318 - 3972,1)^2 \right) = 371076,56\\
    z = \frac{
        \overline{\xb} - \overline{\yb}
    }{
        \sqrt{
            \frac{S^2(x)}{n_x} + \frac{S^2(y)}{n_y}
        }
    }\\
    z = \frac{
        3490,1 - 3972,1
    }{
        \sqrt{
            \frac{544296,56}{10} + \frac{371076,56}{10}
        }
    } = -1,59
\end{gather*}
\begin{align*}
    & P(|z| < \epsilon) = 2\Phi(\epsilon) = 0,95 \\
    & \Phi(\epsilon) = 0,475 \\
    & \epsilon = 1,96
\end{align*}

Поскольку, $|z_{\text{набл}}| < z_{\text{табл}} \Rightarrow |-1,59| < 1,96 \Rightarrow$ гипотеза о равенстве двух <<больших>> выборок принимается.
\newpage

\mysec{Проверка гипотезы о принадлежности <<малых>> выборок к одной генеральной совокупности при помощи критерия медианы.}

Пусть $H_0$ - гипотеза о принадлежности выборки генеральной совокупности.

Найдем медиану:
\begin{align*}
    m = \frac{x_{15} + x_{16}}{2} = 1609,5
\end{align*}

Составим таблицу, где отобразим количество элементов больших или меньших медианы для каждой выборки:
\begin{table}[h!]
    \centering
    \begin{tabular}{|c|c|c|c|c|}
        \hline
        & 1 выборка & 2 выборка & 3 выборка & Всего \\ \hline
        $ > m$ & 0 & 10 & 5 & 15 \\ \hline
        $ < m$ & 10 & 0 & 5 & 15 \\ \hline
        Всего & 10 & 10 & 10 & 30 \\ \hline
    \end{tabular}
\end{table}

Составим таблицу теоретических частот:
\begin{table}[h!]
    \centering 
    \begin{tabular}{|c|c|c|}
        \hline 
        5 & 5 & 5 \\ \hline
        5 & 5 & 5 \\ \hline
        5 & 5 & 5 \\ \hline
    \end{tabular}
\end{table}

Далее найдем $\chi^2$ по формуле:
\begin{align*}
    \chi^2_{\text{набл}} = \sum^2_{i=1}\sum^4_{j=1}\frac{(n_{ij} - m_{ij})^2}{m_{ij}},
\end{align*}
где $n_{ij}$ из таблицы 1, а $m_{ij}$ из таблицы 2.

\begin{gather*}
    \chi^2_{\text{набл}} = 
    \frac{(0 - 5)^2}{5} +
    \frac{(10 - 5)^2}{5} +
    \frac{(5 - 5)^2}{5} +
    \frac{(10 - 5)^2}{5} +
    \frac{(0 - 5)^2}{5} +
    \frac{(5 - 5)^2}{5} = 20 \\
    \chi^2_{\text{табл}}(\alpha = 0,05; df=2) = 5,991
\end{gather*}

Поскольку, $\chi^2_{\text{набл}} > \chi^2_{\text{табл}}$, а значит гипотеза о том, что данные принадлжеат одной генеральной совокупности отклоняется.

\newpage

\mysec{Ранговый критерий Краскела-Уоллиса для <<малых>> выборок}

Пусть $H_0$ - гипотеза о том, что между группами нет разницы. Для начала проранжируем насквозь имеющуюся выборку:
\begin{table}[h!]
    \centering
    \begin{tabular}{|c|c|c|c|c|c|}
        \hline
        Значение & Ранг & Значение & Ранг & Значение & Ранг \\ \hline
        1246 & 8  & 3916 & 21 & 1152 & 2  \\ \hline 
        1144 & 1  & 4262 & 22 & 1300 & 12 \\ \hline 
        1248 & 9  & 4480 & 23 & 1441 & 13 \\ \hline 
        1222 & 6  & 4558 & 24 & 1500 & 14 \\ \hline 
        1199 & 4  & 4638 & 25 & 1595 & 15 \\ \hline 
        1179 & 3  & 4699 & 26 & 1670 & 17 \\ \hline 
        1281 & 10 & 4713 & 27 & 1624 & 16 \\ \hline 
        1201 & 5  & 4912 & 28 & 1801 & 18 \\ \hline 
        1229 & 7  & 5102 & 30 & 1943 & 19 \\ \hline 
        1287 & 11 & 4982 & 29 & 2098 & 20 \\ \hline 
        Сумма рангов & 64 & Сумма рангов & 255 & Сумма рангов & 146 \\ \hline
    \end{tabular} 
\end{table}

Найдем статистику критерия $H$:
\begin{align*}
    H = \frac{12}{N(N+1)} \sum_{i=1}^k n_i \left(\overline{R_i} - \frac{N+1}{2}\right)^2,
\end{align*}
где $N$ - число всех данных, $n_i$ - число данных в $i$-ой группе, $\overline{R_i} = \tfrac{(\sum R_j)_i}{n_i}$.

Получим, что
\begin{gather*}
    H = \frac{12}{60 \cdot 61} \left(
        10\left(6,4 -  \tfrac{60 + 1}{2}\right)^2 + 
        10\left(25,5 - \tfrac{60 + 1}{2}\right)^2 + 
        10\left(14,6 - \tfrac{60 + 1}{2}\right)^2
    \right) = 28,15\\
    \chi^2_{\text{табл}}(\alpha = 0,05; df=2)=5,991
\end{gather*}
Таким образом, $H > \chi^2_{\text{табл}} \Rightarrow$ гипотеза об отсутствии разницы между группами отклоняется.

\newpage
\mysec{Ранговый критерий Фридмана для <<малых>> выборок.}

\end{document}