\documentclass[utf8, a4paper, 14pt, russian, oneside]{book}

% Кодировка
\usepackage[T2A]{fontenc}
\usepackage[utf8]{inputenc}
\usepackage[main=russian, english]{babel}

% Пакеты для работы с математикой
\usepackage{amsmath}
\usepackage{amsfonts}
\usepackage{amssymb}

% Вставка изображений
\usepackage{graphicx}

% Пакет для работы с таблицами
\usepackage{tabularx}
\usepackage{booktabs}
\usepackage{longtable}

% Для больших множеств
\usepackage{mathtools}

% Для работы с рисунками
\usepackage{caption}

% Для создания графов в 3 блоке
\usepackage[all]{xy}

% Для специальных символов
\usepackage{textcomp}
\newcommand{\mysec}[1]{
{\center\section*{#1}}
\addcontentsline{toc}{section}{#1}
}

% Команды для настройки содержания
\renewcommand\contentsname{\center{Содержание}} % Вместо оглавления пишется содержание
\addto{\captionsenglish}{\renewcommand{\bibname}{References}}
\begin{document}

\thispagestyle{empty}
~\vspace{-2cm}\setlength{\parindent}{0cm}
\begin{center}
	\includegraphics[scale=1.5]{../include/logo.png}\\[2pt]
	МИНОБРНАУКИ РОССИИ\\
	Федеральное государственное бюджетное образовательное учреждение\\
	высшего профессионального образования\\[5pt]
	\textbf{<<МИРЭА – Российский технологический университет>>}\\[5pt]
	\textbf{\large РТУ МИРЭА}\\[20pt]
	\hrule{}\mbox{}\\[1pt]
	\hrule{}\mbox{}\\[20pt]	
	Институт кибернетики \\ Кафедра <<Инфориационная безопасность>> (БК №252)\\[35pt]
	\textbf{Долгосрочное задание} \\
	по дисциплине: Математическая статистика
\end{center}
	\vspace{4in}
	Студент группы ККСО-01-19:  \qquad \qquad \qquad  \qquad     Колесников А.В.
\vspace{0.6in}
\begin{center}
Москва --- 2021
\end{center}
\newpage

\tableofcontents
\newpage

\mysec{Описание данных и обоснование разделения на <<малые>> и <<большие>> выборки.}

Для изучения возьмём данные переписи населения в Швейцарии в разные периоды времени (в тыс. человек):
\begin{table}[h!]
    \centering
    \small
    \begin{tabular}{|c|c|c|}
        \hline
        & \textbf{Мужчины} & \textbf{Женщины} \\ \hline
        \multicolumn{3}{|c|}{0-15} \\ \hline
        1980 & 638 & 608 \\ \hline
        1990 & 585 & 559 \\ \hline
        2000 & 642 & 606 \\ \hline
        2002 & 627 & 595 \\ \hline
        2005 & 616 & 583 \\ \hline

        2007 & 606 & 573 \\ \hline
        2009 & 608 & 673 \\ \hline
        2011 & 612 & 589 \\ \hline
        2015 & 631 & 598 \\ \hline
        2019 & 661 & 626 \\ \hline

        \multicolumn{3}{|c|}{16-59} \\ \hline
        1980 & 1951 & 1965 \\ \hline
        1990 & 2148 & 2114 \\ \hline
        2000 & 2250 & 2230\\ \hline
        2002 & 2290 & 2268 \\ \hline
        2005 & 2329 & 2309 \\ \hline

        2007 & 2362 & 2337 \\ \hline
        2009 & 2424 & 2289 \\ \hline
        2011 & 2480 & 2432 \\ \hline
        2015 & 2581 & 2521 \\ \hline
        2019 & 2425 & 2557 \\ \hline

        \multicolumn{3}{|c|}{Больше 60} \\ \hline
        1980 & 482 & 670 \\ \hline
        1990 & 540 & 760 \\ \hline
        2000 & 613 & 828 \\ \hline
        2002 & 641 & 859 \\ \hline
        2005 & 694 & 901 \\ \hline

        2007 & 734 & 936 \\ \hline
        2009 & 774 & 850 \\ \hline
        2011 & 804 & 997 \\ \hline
        2015 & 880 & 1063 \\ \hline
        2019 & 963 & 1135 \\ \hline
    \end{tabular}
\end{table}

Разделим данные на 3 "малые" выборки по возрастной группе: люди от 0 до 15, от 16 до 59 и старше 60 лет.
Поделим данные на 2 "большие" выборки по половому признаку: мужчины и женщины.


\end{document}